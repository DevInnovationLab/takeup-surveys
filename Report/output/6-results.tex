\section{Results}

Table \ref{tab:headlines-vil} shows our main estimates for reach and adoption. Table \ref{tab:headlines-wp} shows the same outcomes, but calculated with the water point as the unit of reference. Table \ref{tab:headlines-am} shows estimates obtained thought the Adoption Monitoring methods.

Table \ref{tab:dsw-func} shows functionality rates for DSW dispensers.
Table \ref{tab:ilc-func} shows functionality rates for ILC dispensers in Malawi.

\subsection{Uganda DSW}

95.1\% of the water points with DSW were functional at the time of the water point census (Table \ref{tab:dsw-func}, column 5), compared to 80.1\% of water points without DSW or ILC. Around 90\% of the dispensers were functional, and 70\% were dispensing chlorine (Table \ref{tab:dsw-func}, column 5). Although similar, all indicators of functionality except for percent of dispensers filled with chlorine lower in Expansion than in Footprint villages (Table \ref{tab:dsw-func-group}. 

Filled dispensers were releasing the expected amount of dilute chlorine solution, at 3 mL. FCR above 0.2 mg/L was detected in water samples treated with chlorine from the dispenser in 66\% of water points. The number is similar for TCR, at 68\%. In both village groups, over FCR was detected at over 95\% of the water points with a filled dispenser. TCR was identified in all of the samples from Expansion villages, and over 95% of those from Footprint villages.


\subsubsection{Malawi DSW}

\subsubsection{Malawi ILC}

\begin{itemize}
            \item 9 villages, served by 19 different water collection points, had no functional ILC water collection points at the time of the visit. \textcolor{red}{These villages are included in the analysis};
            \item \textcolor{red}{6 villages were served by 4 different water point had water collection points fed by a water point where the ILC was not connected to the tank when the visit happened. Since the water points were functional, they were tested for chlorine residual, and these villages are included in our analysis.}
            \item In 1 village, the ILC device served private households in the police compound. Although the community at large does not use the water points with ILC, the field team was able to survey the households that do, and this village is included in our analysis.
            \item In 4 villages included in the analysis, the field team was able to conduct the chlorine residual testing, but access to ILC was more restricted than expected. In 2 of them, because fewer water points were found than included in the administrative data, and in the other two 2, because some of the water points were private.
            \item \textcolor{red}{According to the reports of the village guides, five water points with ILC devices and 26 water collection points were not used by more than one compound.}
\end{itemize}

\subsection{Reach and adoption}

Since are sampling unit is the village, it could be argued that each sampled village, and not household should have the same weight. These two estimates would be similar, since a constant number of households was sampled in each village, but failed tests, attrition, and other survey implementation issues lead the exact number of tests per village to vary. If, however, the goal of this indicator is to present information about the share of all served households in the country that treat their drinking water with chlorine, then villages should be weighed by the number of people using water points with DSW or ILC living in them. Both of these estimates are presented as part of our sensitivity checks.

