\section{Results}

Table \ref{tab:headlines-vil} shows our main estimates for reach and adoption. Table \ref{tab:headlines-wp} shows the same outcomes, but calculated with the water point as the unit of reference. Table \ref{tab:headlines-am} shows estimates obtained thought the Adoption Monitoring methods.

Chlorine Residual detection rates are similar among households whose primary water source has a DSW across both countries and village groups, with 27\% to 33\% of households having TCR levels above 0.2 mg/L according to the color wheel reading, even though the share of dispensers that had chlorine available varies more across the same divisions (between 48\% in Malawi Expansion and 80\% in Uganda in general). If we consider FCR as the main way to measure chlorine presence in the water, detection rates vary between 17\% and 22\%.

Table \ref{tab:dsw-func} shows functionality rates for DSW dispensers.
Table \ref{tab:ilc-func} shows functionality rates for ILC dispensers in Malawi.

\subsection{Uganda}

\begin{itemize}
    \item FCR was detected at 66\% of the water points with DSW.
    \begin{itemize}
        \item 95.1\% of the water points with DSW were functional at the time of the water point census (Table \ref{tab:dsw-func}, column 5), compared to 80.1\% of water points without DSW or ILC. 
        \item Around 90\% of the dispensers were functional, and 70\% were dispensing chlorine (Table \ref{tab:dsw-func}, column 5). Although similar, all indicators of functionality except for percent of dispensers filled with chlorine lower in Expansion than in Footprint villages (Table \ref{tab:dsw-func-group}).
        \item Filled dispensers were releasing the expected amount of dilute chlorine solution, at 3 mL. TCR was detected in 99.2\% of the samples of water treated by the field team at the water point with chlorine from the dispenser, and FCR was detected in 96.1\% (Table \ref{tab:dsw-cl}). 
    \end{itemize}    
    \item We estimate that approximately \textbf{3.8 million people}\textbf{ (95\% CI = 2.9 - 4.8 million)} use water points with dispensers as their primary source of drinking water across all the villages with Dispensers for Safe Water (DSW) (Table \ref{tab:headlines-vil}, Column 4).
    \begin{itemize}
         \item Around 2.9 million of these households live in Expansion villages. These villages are larger than Footprint villages, with an average of 204 households per village compared to 114. They are also more numerous (4,012 vs 3,173), and, on average, a larger portion of the households in these villages use water points with DSW (78\% vs 66\% in Footprint villages). larger number of DSW water points present in each village (4 in Expansion villages compared to 2 in Footprint villages).
         \item We use the village as the main unit of reference for this estimate to best reflect our sampling strategy. Since Evidence Action samples water points instead of villages in their monitoring, we also estimate the same number following this method. Using the water point as the unit of reference, we estimate that 3.5 million people (95\% CI 2.9 - 4.2 million) use water points with DSW across the country (Table \ref{tab:headlines-wp}, Column 4).
         \item Using data from promoters' lists of households using water points with DSW results in a country-level estimate of 4.25 million people (95\% CI = 3.7 - 4.8 million) using water points with DSW.
    \end{itemize}
    \item 42.7/\% of the households using water points with DSW included in the Household Survey reported having treated their drinking water.
    \begin{itemize}
        \item Chlorine dilute solution from dispensers was the most common treatment method, with 30.1\% of household reporting adopting it. 8\% of households reported having boiled the water. 
        \item \textcolor{red}{Include share of households not using DSW that also report treating from the HH census}
    \end{itemize}
     \item Out of the 1,121 households using water points with DSW that had a sample of their drinking water tested using a color wheel, the Total Chlorine Residual (TCR) reading was at least 0.2 ppm in \textbf{30.1\% (95\% CI = 24.5\% - 35.7\%)}, and the Free Chlorine Residual (FCR) in 19.4\% (95\% CI = 14.6\% - 24.1\%) (Table \ref{tab:headlines-vil}, Columns 7 and 8).
    \begin{itemize}
         \item The differences in chlorine detection rates  between Expansion and Footprint villages are not statistically significant, thus we combine the two groups.
         \item TCR was detected in 43.8\% (95\% CI = 36.7\% - 50.8\%) of the households sampled from promoter lists, and FCR in 31.5\% (95\% CI = 25.2 - 37.6\%).
         \item TCR detection rate among households that reported treating the water with chlorine from the dispenser was 64.0\%, compared to 15.2\% among those did not report using this method.
    \end{itemize}
\end{itemize}

\subsection{Malawi}

\subsubsection{DSW}

\begin{itemize}
    \item FCR was detected at 66\% of the water points with DSW.
    \begin{itemize}
        \item Only 2.5\% of the water points with DSW were not functional at the time of the water point census (Table \ref{tab:dsw-func}, column 5), compared to 16.6\% of the water points not served by Evidence Action. 
        \item 83.2\% of the dispensers were functional (the valve could be turned), but only 58.4\% of them were filled.
        \item Dispensers in Expansion villages group had lower functionality rates (74.1\%) and fill rates (48.2\%) than those in Footprint and Expansion villages (both with approximately 90\% of dispensers functional and 65\% filled). Although large, these differences are not statistically significant.
        \item Dispensers were releasing the expected dose of dilute chlorine solution. TCR was detected in 99.5\% of the samples of water treated by the field team at the water point with chlorine from the dispenser, and FCR was detected in 95.6\% (Table \ref{tab:dsw-cl}). 
    \end{itemize}    
    \item We estimate that approximately \textbf{2.1 million people (95\% CI = 1.3 - 2.8 million)} use water points with DSW as their primary source of drinking water in Malawi.
    \begin{itemize}
         \item As in Uganda, most of these households, roughly 1.75 million, live in Expansion villages. There are significantly more Expansion than Footprint villages (4,809 vs 1,161). The average number of households per village is not very different between the two areas, but on average 75\% of the households in Expansion villages use water points with DSW, compared to 50\% in Footprint villages.
         \item Using the water point as the unit of reference, the estimate is of 2 million people (95\% CI 1.5 - 2.5 million).
         \item Using the promoter survey data, we estimate that 3.1 million people (95\% CI 2.8 - 3.4 million) use water points with DSW.         
    \end{itemize}
    \item 32.8\% of the households using water points with DSW included in the Household Survey reported having treated their drinking water.
    \begin{itemize}
        \item Chlorine dilute solution from dispensers was the most common treatment method, with 25.5\% of household reporting adopting it. No other treatment method was used by more than 3\% of households. 
        \item \textcolor{red}{Include share of households not using DSW that also report treating from the HH census}
        \item More households have TCR detected in their drinking water than self-report using chlorine from the dispenser. \textcolor{red}{Double-check that this is not because of "other"methods.} This can be due to the respondent not being responsible for fetching water and therefore not being aware that it had been treated.
    \end{itemize}
     \item Out of the 1,279 households using water points with DSW tested using a color wheel, the TCR reading was at least 0.2 ppm in \textbf{26.6\% (95\% CI = 21.1\% - 32.1\%)}, and the FCR reading in 20.9\% (95\% CI = 16.0\% - 25.7\%).
    \begin{itemize}
     \item The differences in chlorine detection rates  between Expansion and Footprint villages are not statistically significant.
     \item TCR was detected in 30.1\% (95\% CI = 25.4\% - 36.6\%) of the households sampled from promoter lists, and FCR in 27.0\% (95\% CI = 21.6\% - 32.5\%).
     \item TCR detection rate among households that reported treating the water with chlorine from the dispenser was 68.4\%, compared to 15.4\% among those did not report using this method.
    \end{itemize}
\end{itemize}

\subsubsection{ILC}

\begin{itemize}
    \item As described in Section \ref{sec:data-malawi}, non-functionality of ILC water collection points turned out to be so common that we had to adapt our village replacement strategy to not replace villages where all water collection points where out of order.
    \begin{itemize}
        \item Our data includes 9 villages, served by 19 different water collection points, with no functional ILC water collection points at the time of the visit. This is taken into account for reach estimates; 
        \item 6 villages were served by 4 different water points that were not connected to the ILC device when the visit happened. Since the water points were functional, they were tested for chlorine residual, and these villages are included in our analysis.
    \end{itemize}
    \item Even when water collection points were functional, and the ILC device was connected, access to the treated water was often more restricted than expected.
    \begin{itemize}
        \item In 1 village, the ILC device served private households in the police compound. Although the community at large does not use the water points with ILC, the field team was able to survey the households that do, and this village is included in our analysis.
        \item \textcolor According to the reports of the village guides, another five water points with ILC devices and 26 water collection points were not used by more than one compound.
        \item In 4 villages included in the analysis, the field team was able to conduct the chlorine residual testing, but access to ILC was more restricted than expected. In 2 of them, because fewer water points were found than included in the administrative data, and in the other two 2, because some of the water points were private.
    \end{itemize}
    \item \textcolor{red}{Add ILC functionality stats here}
    \item \textcolor{red}{Add ILC chlorination stats here}
    \item We estimate that approximately \textbf{80,000 people}\textbf{ (95\% CI = 50,000 - 100,000)} use ILC water collection points as their primary source of drinking water.
    \begin{itemize}
         \item Using the water point as the unit of reference, we estimate that 72,000 people (95\% CI = 50,000 - 100,000) use ILC water collection points across the country.
         \item Following promoter reports, we estimate that 164,000 people (95\% CI = 135,000 - 195,000) use ILC water collection points across the country.
    \end{itemize}
     \item 389 households using ILC water collection points had a sample of their drinking water tested using a color wheel. Out of these, 28.9\% (95\% CI = 19.1\% - 38.7\%) had TCR of at least 0.2 ppm, and 13.8\% (95\% CI = 8.0\% - 19.5\%) had FCR of at least 0.2 ppm.
     \begin{itemize}
         \item The TCR detection rate is 35.9\% among households using the water collection points closest to the tank where the ILC device is installed, compared to \textbf{35.9\%} among households using other water collection points. FCR detection rates are around 10 percentage points higher among households using the water collection points closest to the ILC device, at 23.1\%.
         \item Although large, these differences are not statistically significant. This is likely due to the small number of households using the water collection points closest to the tank that were tested. Therefore, disaggregated estimates may not be representative of the country-level population.
         \item Among households sampled from those included in promoter lists, 22.3\% (95\% CI = 12.1\% - 32.5\%) had TCR detected in their drinking water, and 14.8\% (95\% CI = 6.3\%	- 23.4\%) had FCR detected.
    \end{itemize}
\end{itemize}

\section{Discussion}

In general, water points with DSW are more likely to be functional than water points with no intervention across both countries, and than water points with ILC in Malawi. This may be due to dispensers being installed at more reliable water points, to water points with dispensers being more frequently maintained, or to dispensers being moved to a different water point when the original installation point stopped functioning. We observed some cases of the latter, but do not have enough data to determine which of these factors are more or less important. Dispenser maintenance and refilling activities seem to vary considerably between countries and village groups, but when filled, the vast majority of dispensers is dosed correctly.
Taken together, the results on functionality and chlorination in ILC water points raise to three potential challenges in ILC operations:

\begin{itemize}
    \item The functionality rate of water points with ILC devices is low when compared to water points with DSW (60\% vs 98\%\footnote{Table \ref{tab:dsw-func}.}).
    \item The fact that FCR detection rates are considerably lower than TCR detection rates indicates that although chlorine was applied to most of the clusters tested, the dosing  was not sufficient for FCR to still be detectable given the chlorine demand in the water and the tank. The water is still likely safer than it would be had it not been chlorinated at all, but it may now be contaminated. 
    \item The difference between the chlorine residual concentration across water collection points receiving water from the same water point with an ILC device, expressed by the different detection rates in the water collection points closer and further from the tank with the device, seems to point to additional chlorine demand in the pipeline or to chlorine decay as the water is distributed through the pipes, so not all the users of a cluster drink similarly safe water. 
\end{itemize}

ILC systems are more complex than DSW, typically requiring a borehole powered by (solar) energy which pumps water into the a tank where the ILC device is installed. Pipes are used to distribute the water from these tanks to taps. Maintaining pumps, solar panels, tanks, pipes, and taps is likely more costly and complicated than maintaining other types of water point where DSW dispensers may be installed, which may explain in part the lower functionality rates at water points with ILC. 

The chlorine residual tests point to the additional challenges of correctly dosing the quantity of chlorine applied to the water by the ILC device. TCR was detected in around 80\% of the clusters of water collection points with water treated by a common ILC device, indicating that most of the ILC devices connected to functional water points are adding chlorine to the water. However, the lower FCR detection rate indicates that the dosing may not be sufficient given the chlorine demands at the different water points. Finally, the variation in chlorine readings throughout the ILC distribution pipeline indicates that, in larger clusters, the same level of protection is not sustained across different water collection points. 

In combination, the lower functionality rate and the challenge of adequately dosing ILC result in a lower availability of treated water to the users of water points with ILC. This may explain the relatively low levels of detectable chlorine residuals among households using water points with ILC, which is very similar to those observed among households using water points with DSw, where water treatment requires the extra step of adding chlorine to the water after collected.

Although the average share of households using water points with DSW/ILC as their primary source of drinking water is similar across countries, at roughly 60\%, this number varies greatly among village groups. In Malawi, approximately 70\% of the 6.6 thousand villages where Evidence Action has dispensers or in-line chlorination devices are in DSW Expansion villages, and this group also has the highest share of households in the villages using DSW/ILC water sources as their primary source of drinking water. Therefore, roughly 85\% (=1.7 million/2 million) of the households estimated to be using water sources with DSW/ILC as their primary source of drinking water in the country are located in these villages.

A small share of households may benefit indirectly from chlorination via secondary sources, but these households are not directly captured in the household survey. Specifically, 45.3\% of households did not use DSW or ILC as their primary source and were thus not eligible for the household survey. Among these, 35.5\% reported access to at least one secondary source, and within this group, 4.3\% reported using chlorine from a dispenser. Taken together, these figures imply that approximately 0.68\% (= 45.3\% × 35.5\% × 4.3\%) could be classified as using water points with a DSW/ILC. Because this share is small and relies on self-reported use of secondary sources, we quantify it separately rather than add it to the headline beneficiary totals.

Figure \ref{fig:people-vil} shows how sensitive results are to the different research decisions described in section \ref{sec:methods-people}. In general, our main point estimates are safely contained in the confidence interval of all alternative estimates. Results in Uganda are most sensitive to the inclusion of households using water points outside of sampled villages, which increases the number of people reached in about 20\%. This is consistent with the observations that villages are located close to one another, and therefore villagers frequently cross village boundaries. In Malawi, where villages are further apart from one another, this decision is of little consequence to the estimates. In this latter country, the most consequential decision is whether we include villages with no functional water collection points in our sample or not. Excluding these villages increases reach estimates by approximately 15\%. 

Figure \ref{fig:chlorine-vil} shows a similar exercise for chlorine detection rate. Here as well, results are comfortably similar. However, in this case we believe that the version of estimates pointing to the lowest level of adoption is just as relevant as our main estimate.

As discussed in Section \ref{sec:methods-cl}, we weigh all tested households equally because it is our understanding that this the most comparable measure to Evidence Action's estimates. Since are sampling unit is the village, each sampled village, and not each household should have the same weight. These two estimates would be similar, since a constant number of households was sampled in each village, but failed tests, attrition, and other survey implementation issues lead the exact number of tests per village to vary. If, however, the goal of this indicator is to present information about the share of all served households in the country that treat their drinking water with chlorine, then villages should be weighed by the number of people using water points with DSW or ILC living in them.

The results suggest that self-reports of water treatment with chlorine are consistent with color wheel measurements, although the answers do not always align at the household level. Chlorine is detected in around 66.2\% of the samples from households that reported using it to treat the water tested. While this figure is substantial, it does not approach 100\%, due to two potential reasons: chlorine decay over time and the over-reporting, as respondents may feel social pressure to affirm chlorine use when asked by enumerators. Among households that did not report chlorine use, the chlorination rate is about 12.5\% percent. When testing untreated water at the water points, we observed a false positive rate of about 2\% with a color wheel, which indicates that respondents may not have been aware of whether their water was treated with chlorine or not. The chlorine residual detection rates among households that did not report using chlorine are higher when measured with a colorimeter, with detection levels ranging from 8\% to 32\%. Further investigation is needed to understand the potential of false positives, such as the presence of certain minerals in the water that may oxidize with the reagent and produce higher-than-expected chlorination readings.

Self-reported treatment is a better predictor for chlorine residual detection than using a water point with a dispenser to collect the water tested. Depending on the village group, the chlorine residual detection rate is not different between people who reported the tested water came from their water point (which we know to have a DSW) and people who reported using a secondary water source. This is partially due to the fact that we do not have data on whether secondary water sources also have chlorine dispensers. But anecdotal evidence suggests that some households also receive bottled water from the promoters or use chlorine from the dispenser even when fetching water from a different source.

