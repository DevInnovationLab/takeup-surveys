\section{Results}\label{sec:results}

This Section summarises and discusses our results, shown in Tables \ref{tab:headlines-vil} to \ref{tab:ilc-cl}. We estimate that approximately 3.9 million people in Uganda use water points with DSW as their primary source of drinking water. In Malawi, we estimate this number to be about 2 million people, and that another 75,000 people use water points with ILC as their primary source of drinking water.

\textcolor{blue}{\textit{TCR detection rates are similar among households whose primary water source has a DSW across both countries and village groups, with 27\% to 33\% of households having TCR levels above 0.2 mg/L according to the color wheel}}. This is despite the share of dispensers that had chlorine available varying across the same divisions (between 48\% in Malawi Expansion and 80\% in Uganda in general). If we consider FCR as the main way to measure chlorine presence in the water, detection rates vary between 17\% and 22\%. \textit{TCR detection rates among households using water points with ILC (29\%) are similar to those among households using DSW}. This can be rationalized by taking into account that while ILC does not require the additional user effort of water treatment or the discomfort of handling dilute chlorine, it may be harder to dose and maintain. 

\textcolor{blue}{The latter claim is supported by findings from our water point census, shown in Tables \ref{tab:dsw-func} to \ref{tab:ilc-cl}. More than 95\% of the water points with DSW were functional at the time of the survey, and we detect FCR concentrations above 0.2 mg/L in 96\% of the water points with filled and functional dispenser. In ILC systems, on the other hand, both functionality and FCR detection rates are lower: around 85\% of the ILC water collection points were functional at the time of the survey, and FCR was detected in water samples from 64\% of the water collection points closest to the ILC device and 26.5\% of the other water collection points tested.}

\textit{Estimates obtained using the village and the water point as units of reference are very similar.} This points to a water-point based sampling method, which is generally cheaper, being a viable option. \textit{Using Adoption Monitoring methods, on the other hand, leads to more contrasting results.} In Uganda, the estimated reach is similar to that in Tables \ref{tab:headlines-vil} and \ref{tab:headlines-wp}, but chlorination rates, at 46\% for TCR and 33\% for FCR, are higher. In Malawi, on the other hand, chlorination rates are comparable to those obtained using our preferred method, but estimates of reach are higher by about 50\% for DSW, and 185\% for ILC. Sections \ref{sec:adoption} and \ref{sec:reach} explore potential explanations for this.

\subsection{Discussion}

\subsubsection{Functionality and access to chlorine}

\textit{In both countries, water points with DSW are more likely to be functional than water points with no intervention; in Malawi, the latter are also more likely to be functional than water points with ILC.} This may be due to dispensers being installed at more reliable water points, water points with dispensers being more frequently maintained, or dispensers being moved to a different water point when the original installation point stops functioning. We observed some cases of the latter, but do not have enough data to determine the relative importance of these factors. Dispenser maintenance and refilling activities seem to vary considerably between countries and village groups, but when filled, the vast majority of dispensers are dosed correctly.

Taken together, the results on functionality and chlorination in ILC water points point to two potential challenges in ILC operations.

Among the water collection points located the closest to the water point feeding the cluster, we detected TCR in 82\%, indicating that the water in the tank was treated. On the other hand, we only detected FCR in 64\% of these water collection points, meaning the dose was inadequate given the chlorine demand in the water and the tank. The other water collection points in the system show even lower TCR and FCR detection rates, at 37\% and 26.5\%, respectively. This indicates that there is additional chlorine demand in the system that also needs to be addressed by the dosing.

\textcolor{blue}{Since functionality rates and Chlorine Residual detection at the water point do not point to any major issues with dilute chlorine solution at water points with DSW, we attribute the existing adoption rates primarily to household preferences and the behavioral component of dispenser usage. Issues with refilling of dispensers may also contribute to this. However, the fact TCR is detected in the water samples from 9\% to 30\% of households using water points with empty dispensers, depending on the sample group (\textcolor{red}{Add table}) indicates that this is unlikely to be the main driver.\footnote{The sample size for these adoption estimates is considerably smaller than for households using water points with filled dispensers.}}


First, \textit{the lower functionality rate of ILC water points may be due to ILC systems being complex than DSW}. Maintaining pumps, solar panels, tanks, pipes, and taps is likely more costly and complicated than maintaining other types of water point where DSW dispensers may be installed.\footnote{Remember that Evidence Action is not responsible for installing or maintaining the water points themselves, just for the ILC device that is connected to existing infrastructure.} A variety of reasons for non-functionality of these water points was mentioned, including broken pumps, broken solar panels, and leaking tanks. In one of the water points where the ILC device was disconnected, the operator also indicated that the connection to the device reduced the water pressure. Nevertheless, given the relatively small number of ILC water points in our sample, this observation may not be as consequential as the statistics suggest. \textcolor{blue}{Additionally, water collection points connected to functional ILC water points had a high functionality rate\textcolor{red}{(CITE NUMBER)}, indicating that this problem may be concentrated among specific clusters and operators.}

\textit{The Chlorine Residual tests point to the additional challenge of correctly dosing the quantity of chlorine applied to the water by the ILC device.} TCR was detected in around \textbf{80\%} of the clusters of water collection points with water treated by a common ILC device, indicating that most of the ILC devices connected to functional water points are adding chlorine to the water. However, the lower FCR detection rate indicates that the dosing may not be sufficient given the chlorine demands at the different water points. In such cases, the water is likely safer than it would be had it not been chlorinated at all, but it may still be contaminated. Similarly, the variation in chlorine readings throughout the ILC distribution pipeline indicates that, in larger clusters, the same level of protection is not sustained across different water collection points. 

\textbf{In combination, the lower functionality rate of water points and the challenge of adequate dosage result in a lower availability of treated water to users of water points with ILC.} This may explain the relatively low level of detectable chlorine residuals among households using water points with ILC, with TCR levels very similar to those observed among households using water points with DSW, where water treatment requires the extra step of adding chlorine to the water after collection.

\textbf{Still, maintaining of a limited number of water points may be a simpler problem than than promoting behavioral change.} Since functionality rates and detection at the water point do not point to issues with chlorine availability in dispensers, we attribute the existing adoption rates primarily to household preferences and the behavioral component of dispenser usage. The comparison of self-reported water treatment practices between household using water points with and without DSW is consistent with this mechanism. Keep in mind, though, that the questions answered by the two groups are not the same: households using water points DSW were asked about whether they treated the water that was going to be tested, while the other households were asked about their \textit{usual} water treatment practices.

In Uganda, households using water points with DSW were just are likely to report treating the water in the sample as households using other water points usually treat theirs, but households with water points with DSW were twice as likely to using chlorine for treatment, at around 30\%, as household using other water points, at approximately 15\%. In Malawi, on the other hand, the share of households that reported treating the drinking water with chlorine is similar among the two groups (about 30\%), but households using water points with DSW are three times more likely to use the dilute chlorine solution from the dispenser, at 30\%, as those using other water points, at 10\%. Households using water points with ILC are less likely to report treating the water than any other group in our sample.

\subsubsection{Choice of water point}

\textbf{The majority of households using water points with DSW/ILC — about 80\% — relied on a single water point as their primary source of drinking water} for the month before the interview, while the remaining 20 percent report using on average 2.2 water points. Prevalence of use of secondary water sources among these households varies across village categories, from 14\% in Expansion villages of Malawi (which also has the lowest average number of water points per village) to roughly 24\% in Footprint villages in Uganda as well as Footprint and ILC villages in Malawi. 

\textbf{In Malawi, households using water points with ILC are the most likely to use more than one water point,} which is consistent with the lower functionality rate of these water points. However, their are not significantly more likely than households using water points not served by DSW/ILC to do so. Households using water points with DSW are the least likely to have collected drinking water from more than one water collection point in the month before data collection.

\textbf{Water samples for chlorine testing were drawn mainly from the households' primary water point}, with the share of such cases ranging from 85\% in Uganda Footprint villages to 96\% in Malawi Expansion villages. A smaller share reported collecting water from another point inside the village (2–9\%), from outside the village (2–4\%), or from rainwater or piped sources (each under 3\%). These patterns indicate that the testing data largely reflect the water sources households rely on most.

\textbf{A small share of households may benefit indirectly from chlorination via secondary sources, but these households are not directly captured in the Household Survey.} Specifically, 45.3\% of households did not use DSW or ILC as their primary source and were thus not eligible for the Household Survey. Among these, 35.5\% reported access to at least one secondary source, and within this group, 4.3\% reported using chlorine from a dispenser. Taken together, these figures imply that approximately 0.68\% (= 45.3\% × 35.5\% × 4.3\%) could be classified as using water points with a DSW/ILC. Because this share is small and relies on self-reported use of secondary sources, we quantify it separately rather than add it to the headline beneficiary totals.

\subsubsection{Sensitivity checks}

Figure \ref{fig:users-vil} shows how sensitive results are to the different research decisions described in section \ref{sec:methods-people}.\textbf{In general, our main point estimates are safely contained in the confidence interval of all alternative estimates.} Results in Uganda are most sensitive to the inclusion of households using water points outside of sampled villages, which increases the number of people reached by about 20\%. This is consistent with the observations that villages are located close to one another, and therefore villagers frequently cross village boundaries. In Malawi, where villages are further apart from one another, this decision is of little consequence to the estimates. In this country, the most consequential decision is whether we include villages with no functional water collection points in our sample or not. Excluding these villages increases reach estimates by approximately 15\%. 

Figure \ref{fig:chlorine-vil} shows a similar exercise for chlorine detection rates. Here as well, results are comfortably similar. As discussed in Section \ref{sec:methods-cl}, we weigh all tested households equally because it is our understanding that this the most comparable measure to Evidence Action's estimates. Since our sampling unit is the village, each sampled village, and not each household, should have the same weight. These two estimates would be similar, since a constant number of households was sampled in each village, if not for failed tests, attrition, and other survey implementation issues, which lead the exact number of tests per village to vary. If the goal of this indicator is to present information about the share of all served households in the country that treat their drinking water with chlorine, then villages should be weighed by the number of people using water points with DSW/ILC living in them. Therefore, \textbf{we would argue that, in the case of adoption, the version of estimates pointing to the lowest level of adoption is just as relevant as our main estimate.}

\subsubsection{Self-reported and objective measures of chlorination}



\textbf{In our data, self-reports of water treatment with chlorine are consistent with color wheel measurements, although the answers do not always align at the household level.} Chlorine is detected in around 66.2\% of the samples from households that reported using it to treat the water tested. While this figure is substantial, it does not approach 100\%, due to two potential reasons: chlorine decay over time and over-reporting, as respondents may feel social pressure to affirm chlorine use when asked by enumerators. Among households that did not report chlorine use, the chlorination rate is about 12.5\% percent. When testing untreated water at the water points, we observed a false positive rate of about 2\% with a color wheel, which indicates that respondents may not have been aware of whether their water was treated with chlorine or not. The chlorine residual detection rates among households that did not report using chlorine are higher when measured with a colorimeter, with detection levels ranging from 8\% to 32\%. Further investigation is needed to understand the potential of false positives, such as the presence of certain minerals in the water that may oxidize with the reagent and produce higher-than-expected chlorination readings.

\textbf{Self-reported treatment is a better predictor for chlorine residual detection than using a water point with a dispenser to collect the water tested.} Depending on the village group, the chlorine residual detection rate is not different between people who reported the tested water came from their water point (which we know to have a DSW) and people who reported using a secondary water source. This is partially due to the fact that we do not have data on whether secondary water sources also have chlorine dispensers. But anecdotal evidence suggests that some households also receive bottled water from the promoters or use chlorine from the dispenser even when fetching water from a different source.

\textit{Adoption estimates obtained using the village and the water point as units of reference are extremely similar.} This was expected, since they are effectively taking unconditional averages of almost the same sets of households. \textit{Reach estimates are also similar, but lower when using the water point. The lower estimates may be explained by the fact the  which is expected given that the ``catchment area'' is incomplete} -- that is, some of the water points are likely to have users that live more than 200 meters outside of the village boundaries and therefore are not included in our survey.  This points to a water-point based sampling method, which is generally cheaper, being a viable option.
