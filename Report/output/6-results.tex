\section{Discussion of main results}\label{sec:results}

\textit{Table  \ref{tab:headlines-vil} shows our main estimates of reach and adoption of Evidence Action interventions.} We estimate that approximately four million people in Uganda use water points with DSW as their primary source of drinking water (column 4). In Malawi, we estimate this number to be about two million people, and that another 75,000 people use water points with ILC as their primary source of drinking water. TCR detection rates are similar among households whose primary water source has a DSW across both countries and interventions, with 26.4\% to 30.7\% of households having TCR levels above 0.2 mg/L according to the color wheel (column 7).\footnote{Households using DSW water points in ILC villages show a lower TCR detection rate, at 20.6\% (Table \ref{tab:headlines-vil-by-group}).} Since most of the households only use one water point, we believe that our data largely reflects the water sources households rely on most. With around 90\% of the tested samples being collected at households' primary water sources, which we have verified to have DSW/ILC, we believe our estimates to also indicate the levels of chlorine treatment adoption in the program.

The similar chlorine detection rates across interventions may be explained by the fact that while ILC does not require additional steps from users to treat the water after fetching it and avoids the discomfort of handling dilute chlorine, ILC systems can be harder to dose and maintain. Findings from our water point census, shown in Tables \ref{tab:dsw-func} to \ref{tab:ilc-cl}, are consistent with this claim. More than 95\% of the water points with DSW were functional at the time of the survey, 70\% had a functional and filled dispenser, and we detected FCR concentrations above 0.2 mg/L in 96\% of the water points with a filled and functional dispenser. \textit{Overall, FCR was detected in 60\% of the water points with DSW.} In ILC systems, on the other hand, both functionality and FCR detection rates are lower: around 85\% of the ILC water collection points were functional at the time of the survey, and \textit{FCR was detected in water samples from 64\% of the water collection points closest to the ILC device and 26.5\% of the other water collection points that were tested. }

Estimates obtained using our preferred method are very similar whether we use the village (Table \ref{tab:headlines-vil}) or the water point (Table \ref{tab:headlines-wp}) as units of reference. This points to a water-point based sampling method, which is generally cheaper, being a viable option. On the other hand, a method that tries to approximate Evidence Action's Adoption Monitoring estimates, shown in Table \ref{tab:headlines-am}, leads to more contrasting results. In Uganda, estimated reach is similar across all three methods, but estimates of adoption following the Adoption Monitoring method are almost 50\% higher than those of the other two methods. In Malawi, on the other hand, adoption estimates are comparable to or lower than those obtained using our preferred method, but reach estimates are higher by about 50\% for DSW, and more than 200\% for ILC. 

The remainder of this Section discusses the results mentioned above. A bullet-point summary of results discussed in this Section can be found in Appendix \ref{sec:appendix-results}.

\subsection{Reach and adoption}

\textbf{We estimate that 3.9 million people (95\% CI = 2.9 - 4.8 million) use water points with DSW in Uganda.} Around 2.9 million of these live in Expansion villages.\footnote{Table \ref{tab:headlines-vil-by-group}.} These villages are larger than Footprint villages, with an average of 203.5 households per village compared to 114. They are also more numerous (4,051 vs 3,365), and, on average, a larger portion of the households in these villages use water points with DSW (79\% vs 65.6\% in Footprint villages). On average, there are also more water points with DSW in per village in the Expansion group (4.1) than in the Footprint group (2.0). Out of the 1,121 households using water points with DSW that had a sample of their drinking water tested using a color wheel, \textbf{TCR above 0.2 mg/L was detected in \textbf{30.4\% (95\% CI = 24.8\% - 36.1\%)}, and the FCR in 19.6\% (95\% CI = 14.9\% - 24.3\%)}.\footnote{Table \ref{tab:headlines-vil}, columns 7 and 8.}}

\textbf{In Malawi, we estimate that approximately \textbf{2 million people (95\% CI = 1.3 - 2.7 million)} use water points with DSW as their primary source of drinking water, \textcolor{red}{and .}} As in Uganda, most of these people, roughly 1.6 million, live in Expansion villages. There are significantly more Expansion than Footprint villages (4,403 vs 1,567). The average number of households per village is not very different between the two sample groups, but on average 75\% of the households in Expansion villages use water points with DSW, compared to 50\% in Footprint villages. \textbf{Out of the 1,279 households using water points with DSW tested using a color wheel, the TCR reading was at least 0.2 mg/L in 26.4\% (95\% CI = 21.0\% - 31.9\%), and the FCR reading in 20.7\% (95\% CI = 15.9\% - 25.5\%).} 

\textit{All alternative estimates obtained when changing the methodological decisions described in Section \ref{sec:methods} are within the confidence interval or our main estimates, although reach estimates in Uganda do vary considerably}, increasing by up to 20\%. Figures \ref{fig:users-vil} and \ref{fig:chlorine-vil} shows the results of sensitivity checks for our estimates of reach and adoption, respectively. All results in these figures are based on variations of the methods used in Table \ref{tab:headlines-vil}, relying on self-reports of primary water source from the Household Census and using water tests from households in the census sample of the Household Survey.

\textit{In Malawi, point estimates of reach are relatively stable under these sensitivity checks.} DSW point estimates range from 1.98 to 2.26 million people, that is, from a 2\% decrease to a 12\% increase over our main results. ILC point estimates range from 68 to 79 thousand people, or from a 10\% decrease to a 5\% increase. 

\textit{Reach estimates in Uganda, on the other hand, vary considerably: they range from 3.4 to 4.7 million people, a 12\% decrease and a 20\% increase, respectively. }The decisions on whether households and water points outside of village boundaries are included in our estimates, as well as whether or not to attempt a correction for non-response rates, are the most consequential. It is unsurprising that this should be the case. Given the proximity between villages, the use of village boundaries to establish a ``catchment area'' for water points partially fails its purpose of being an organic delimitation of the area around which households move. However, given that we use the number of villages in the country to extrapolate our estimates to the national level, including water points outside of sampled villages in the sample would lead to double-counting of households. Similarly, the relatively large non-response rates in Uganda mean that this adjustment has a larger impact in this country than in Malawi. Although our strategy of assuming similar use of water points with DSW among households observed and not observed in our data is an imperfect correction, we believe it is a necessary one, especially in light of anecdotal evidence pointing to unrecorded non-responses.

\textit{\textcolor{red}{Alternative adoption point estimates vary from a 11\% decrease to a 15\% increase.} Unlike with reach estimates, however, in this case we believe there is good reason to use one alternative estimate in particular: the one that weighs villages by the number of DSW/ILC users in them. This would point to adoption rates of 23.5\% for households reached by DSW in Malawi, 26.9\% for households reached by ILC in Malawi, and 27.4\% for households reached by DSW in Uganda.} \textcolor{red}{As discussed in Section \ref{sec:methods-cl}, in our main estimates we weigh all tested households equally because it is our understanding that this the most comparable measure to Evidence Action's estimates. However, since our sampling unit is the village, if we wanted to calculate an unweighted mean, then each village, not each household, should have the same weight. This would make little difference, since a constant number of households was sampled in each village, if not for failed tests, attrition, and other survey implementation issues, which lead the exact number of tests per village to vary. Crucially, however, if the goal  of this indicator is to present information about the share of all reached households in the country that treat their drinking water with chlorine, then even an unweighted average of villages would not be appropriate. To obtain such an indicator, villages would have to additionally be weighed by the number of people using water points with DSW/ILC living in them.}

\subsection{Access to chlorine at DSW water points}

Setting aside household-specific water collection practices that may lead to lower or higher chlorine decay in treated water, two main factors work in tandem to determine chlorine treatment adoption rates among households using water points with DSW. First, the availability of chlorine to treat the water. Dispensers make chlorine easily available at the water points, but households may also obtain it elsewhere. Second, the household's water treatment practices, that is, whether its members have the custom of treating the drinking water with chlorine. These factors naturally influence one another, as availability enables households to develop the habit of treating water, and this habit may lead households to procure chlorine when it is not available. This Section explores how much each of them can explain the observed chlorine detection rates. 

\textit{In both Malawi and Uganda, over 95\% of the water points with DSW are functional, compared to roughly 80\% of water points without DSW/ILC.} This may be due to dispensers being installed at more reliable water points, water points with dispensers being more frequently maintained, or dispensers being moved to a different water point when the original installation point stops functioning. We observed some cases of the latter, but do not have enough data to determine the relative importance of these factors. Either way, the fact remains that households can probably rely on these water points consistently.

\textit{Dispenser maintenance and refilling activities seem to vary considerably between countries and village groups.} At least 94\% of dispensers installed across countries and village groups were functional (the tank was present and the valve could be turned), with the exception of Expansion villages in Malawi, where the functionality rate was 86\%. In Uganda, 78\% of the functional dispensers were filled. In Malawi,  72\% and 76\% of the functional dispensers were filled in ILC and Footprint villages, respectively, compared to only 65\% of those in Expansion villages. When filled, the vast majority of dispensers are dosed correctly, and chlorine was detected in 96\% of the water samples tested after treatment with dilute chlorine solution from a dispenser. Overall, FCR was detected in 66\% of water points with DSW in Uganda, and in Malawi, in 60\% of water points with DSW in Footprint and ILC villages and 46\% of the water points with DSW in Expansion villages. Therefore, dilute chlorine availability can at times be limited by inconsistent promoter activities. \textcolor{red}{That said, even the lowest shares of FCR detection at the water point are over double the highest observed adoption rates at households in our main estimates, suggesting that this constraint is not binding for household chlorination.}

\textit{Moreover, TCR was detected in the water samples from 9\% to 30\% of households using water points with empty dispensers,} depending on the sample group and country (\textcolor{red}{Add table}),\footnote{The sample size for these adoption estimates is considerably smaller than for households using water points with filled dispensers.} which indicates that they are able to access chlorine (for example, from a dispenser at a non-primary water point) when their primary dispenser is not filled. 

\textbf{\textit{Our results indicate that household preferences and the behavioral component of dispenser usage likely sets the upper bound for adoption rates.} }Additional support for this explanation can be drawn from self-reported water treatment. In Uganda, households using water points with DSW were just as likely to report treating \textcolor{red}{(using at least one of a range of methods)} the water \textcolor{red}{in the tested sample} as households using other water points; however, households using water points with DSW were twice as likely to use chlorine for treatment, at around 30\%, as households using other water points, at approximately 15\%. In Malawi, on the other hand, the share of households that reported treating drinking water with chlorine is similar among the two groups (about 30\%), though households using water points with DSW are three times more likely to use dilute chlorine solutio, n from a dispenserat 30\%, as those using other water points, at 10\%. 

\subsection{Access to treated water in ILC systems}

In the water points census, around 85\% of the water points in ILC systems were functional, \textcolor{red}{compared to 97\% of water points with DSW} and 80\% of non-intervention water points in ILC villages. Of the 25 water points with a tank in ILC clusters observed in our data, 92\% had an ILC device present during the water point census, and 89\% of the 18 ILC devices for which field officers could ascertain functionality were working. Note that these sample sizes are considerably smaller than those observed in DSW villages, and therefore are more noisy and more likely to suffer from small sample bias.

We observe 176 ILC water collection points. Seven have mostly missing data (see Section \ref{sec:data-malawi} for a discussion of why), and 85.6\% of those where functionality was assessed were working. We tested 11 water collection points that were the first water points in the pipeline and therefore less likely to be affected by chlorine decay in the distribution system. Out of these, 82\% had TCR detected, indicating that the majority of water points was receiving water that had been treated with chlorine, but FCR was only detected in 64\%, indicating that the dosing was likely inadequate given the chlorine demand in the water and the tank. We also tested 49 water collection points that were further away from the ILC device, and detected TCR and FCR in 37\% and 26.5\% of them, respectively. 

Taken together, the results on functionality and chlorination in ILC water points point to two potential challenges in ILC operations.

First, \textit{the lower functionality rate of ILC water points may be due to ILC systems being complex than DSW}. Maintaining pumps, solar panels, tanks, pipes, and taps is likely more costly and complicated than maintaining other types of water points (e.g., boreholes, dug wells) where DSW dispensers may be installed.\footnote{Remember that Evidence Action is not responsible for installing or maintaining the water points themselves, just for the ILC device that is connected to existing infrastructure.} A variety of reasons for non-functionality of these water points was mentioned, including broken pumps, broken solar panels, and leaking tanks. In one of the water points where the ILC device was disconnected, the operator also indicated that the connection to the device reduced the water pressure. Nevertheless, given the relatively small number of ILC water points in our sample, this observation may not be as consequential as the statistics suggest. Additionally, water collection points connected to functional ILC water points had a high functionality rate\textcolor{red}{(CITE NUMBER)}, indicating that this problem may be concentrated \textcolor{red}{among specific clusters or operators; one explanation for this could be that promoters may have less incentive to maintain a device if they know that the associated collection point or points are not functional.}

\textit{Second, the Chlorine Residual tests point to the additional challenge of correctly dosing the quantity of chlorine applied to the water by the ILC device.} The low FCR detection rates across water points compared with TCR rates at the closest water point to the ILC device indicate that when chlorine is being added, the dosing may not be sufficient given the chlorine demands at the different water points. In such cases, the water is likely safer than it would be had it not been chlorinated at all, but it may still be contaminated. Similarly, the variation in chlorine readings throughout the ILC distribution pipeline indicates that, in larger clusters, the same level of protection is not sustained across different water collection points. 

\textbf{In combination, the lower functionality rate of water points and the challenge of adequate dosage result in a lower availability of treated water to users of water points with ILC.} This may explain the relatively low level of detectable chlorine residuals among households using water points with ILC, with TCR levels very similar to those observed among households using water points with DSW, where water treatment requires the extra step of adding chlorine to the water after collection.


\subsection{Use of secondary water points and source of tested water}

\textit{The majority of households using water points with DSW/ILC (79.5\%) relied on a single water point as their primary source of drinking water} for the month before the interview, while the remaining 20\% report using on average 2.2 water points. \textbf{Therefore, we believe that our data largely reflects the water sources households rely on most}. 

\textit{Households using water points with ILC are the most likely to use more than one water point,} with 33.5\% of them reporting having done so in the month preceding the survey, compared to 19.6\% of households using water points with DSW and 26.8\% of households whose primary water point does not have DSW/ILC. This finding is consistent with the lower functionality rate observed among water points with ILC.

\textit{Although some households may benefit indirectly from chlorination via secondary sources, our data indicates that this is a negligible share.} 35.4\% of households in the Household Census did not use a water point with DSW/ILC within their village as their primary water source and were thus not eligible for inclusion in the Household Survey. Among these, 26.5\% reported using at least one secondary source. Therefore, up to 9.4\% (26.5\% $\times$ 35.4\%) of the households in our sample \textit{may} use a water points with DSW/ILC as a secondary water source. Within this group, 7.7\% reported using chlorine from a dispenser, meaning that from our data we have evidence that 1\% ($\approx$ 9.4\%  $\times$ 7.7\%) of the households not included in our reach estimates may actually be reached by the programs. Given this indication that reach through secondary water sources is small and the limited data on secondary water point usage, we do not attempt to include these households in our estimates.

\textit{Water samples for chlorine testing were drawn mainly from the households' primary water point}. Overall,  89.3\% of all households tested reported collecting the water from their primary source of drinking water. This share varies across countries and interventions: in Uganda,  it represents 85.8\% of tested households, compared with 93.5\% of tested households using water points with DSW in Malawi, and 84.4\% of households using water points with ILC. A smaller share of households reported collecting water from another point inside the village (6.3\%), from outside the village (2.8\%),  from rainwater (1\%), or other sources (0.8\%). \textbf{From this we conclude that our estimates are indicative of the level of chlorine adoption in the program.}

\subsection{Measuring chlorine treatment adoption}

We observe seven different measures of chlorine treatment adoption in our data, summarized in Tables \ref{tab:cl-criteria-country} and \ref{tab:cl-criteria-sample}.We collect self-reported data, color wheel readings for TCR and FCR, and colorimeter readings TCR and FCR. Color wheels have a minimum detection threshold of 0.2 mg/L, and colorimeters have a minimum detection threshold of 0.1 mg/L. We calculate colorimeter chlorine detection rates at 0.1mg/L to make use of the higher granularity of this instrument and at 0.2mg/L to have a comparable measure to color wheels. Therefore, apart from self-reports, we have two different indicators from color wheel readings and four from colorimeter readings.

\textit{As discussed in Section \ref{sec:methods-cl}, measurements obtained with a colorimeter presented inconsistencies that led us to disregard them for the purpose of this report.} Chlorine detection rates obtained with a colorimeter are presented in Appendix \ref{sec:appendix-cl}, but we recommend not relying too heavily on them. Further investigation is needed to understand the potential of false positives, such as the presence of certain minerals in the water that may oxidize with the reagent and produce higher-than-expected chlorination readings.

\textit{In our data, self-reports of water treatment with chlorine are consistent with color wheel measurements, even though the two do not always align at the household level.} TCR is detected by the color wheel in around 67\% of the samples from households that reported using it to treat the water tested. While this figure is substantial, it does not approach 100\%, due to two potential reasons: chlorine decay over time and over-reporting, as respondents may feel social pressure to affirm chlorine use when asked by enumerators. 

\textit{Among households that did not report treating the water with chlorine, the chlorine detection rate is about 15.5\% percent.} When testing untreated water at the water points, we observed a false positive rate of about 2\% with a color wheel, which indicates that respondents may not have been aware of whether their water was treated with chlorine or not.

%\textcolor{red}{Self-reported treatment is a better predictor for chlorine residual detection than using a water point with a dispenser to collect the water tested. Depending on the village group, the chlorine residual detection rate is not different between people who reported the tested water came from their water point (which we know to have a DSW) and people who reported using a secondary water source. This is partially due to the fact that we do not have data on whether secondary water sources also have chlorine dispensers. But anecdotal evidence suggests that some households also receive bottled water from the promoters or use chlorine from the dispenser even when fetching water from a different source.}

\subsection{Comparison of estimation methods}

Figures \ref{fig:reach-comparison} and \ref{fig:adoption-comparison} compare the reach and adoption estimates obtained with each of the three estimation methods described in Section \ref{sec:methods}.

\textit{Adoption estimates obtained using the village and the water point as units of reference are extremely similar}. This was expected, since they are effectively taking unconditional averages of almost exactly the same set of households. \textit{Reach estimates are also similar, but lower when using the water point. The lower estimates may be explained by the fact that the ``catchment area'' is incomplete in the village case} --  some of the water points are likely to have users that live more than 200 meters outside of the village boundaries and therefore are not included in our survey. \textbf{This points to a water-point based sampling method, which is generally cheaper, being a viable option.}

\textit{By contrast, methods that try to approximate Evidence Action's Adoption Monitoring estimates lead to very different results from our main estimates.} In Uganda, the results on number of people reached are similar across all three methods, but TCR-based estimates of adoption following the Adoption Monitoring method are almost 50\% higher than those obtained using the other two methods, and FCR-based ones are 100\% higher. In Malawi, on the other hand, reach estimates are higher by about 50\% for DSW, and more than 200\% for ILC; and adoption estimates among households using DSW are comparable to those obtained using our preferred method, as are FCR-based estimates among households using ILC, but TCR-based adoption estimates are 50\% lower. 

\textcolor{red}{
The differences in ILC estimates seem to be related to
    (i) promoters listing households who may use ILC, but not as their primary water source. Among all water points in the promoter survey, the number of users listed by the promoter is 3.5 times the numer of self-reported users. According to promoters, only X\% of these users are from outside the village, and in our data we see Y\%. We know that households using ILC water points as their primary source of drinking water are more likely to use other water points as well, and hypothetize that this is connected to lower ILC functionality rates. Conversely, it is likely that households that have a different primary water source also use ILC as a secondary source when it is functional. The promoter survey would include these households, but not the census sample. Annecdotally, we talked to households that were listed by the promoter but only used the ILC water point when they had the monay to do it. Only 70\% of the households in the monitoring survey confirmed that the ILC water point is their primary source of drinking water, compared to 83\% in Malawi DSW and 90\% in Uganda DSW.}

\textcolor{red}{
    (ii) None of the households in the monitoring survey that said the ILC water point is not their primary source of drinking water had chlorine detected. water points with ILC included in the promoter survey where 17\% less likely to have TCR detected -- NEED to look at cluster level instead of WP.
}

\textcolor{red}{In DSW, the explanation is not as straight forward.} Sections \ref{sec:adoption} and \ref{sec:reach} explore potential explanations for the differences observed.