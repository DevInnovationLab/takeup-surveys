\section{Exploration of differences in reach estimates}\label{sec:reach}

This Section discusses potential explanations for the significant differences observed in estimates of the number of people using water points with DSW/ILC resulting from the census (Table \ref{tab:headlines-vil}) and the Adoption Monitoring (Table \ref{tab:headlines-am}) methods. Section \ref{sec:reach-list} lists a series of factors that may contribute to this differences. Section \ref{sec:reach-waterfall} discusses the estimates, shown in Figures \ref{fig:waterfall-mw} and \ref{fig:waterfall-ug}. The charts allow us to discount some hypotheses, which either explain very little, or indeed point in the opposite direction to the overall difference. Our interpretation is that differences are primarily driven by selection of water points and, in the case of Malawi, by households located outside of the village. However, this analysis is not conclusive: in Malawi, where the difference in estimates was larger to begin with (approximately 50\%), the potential explanations whose impact we are able to quantify leave much of the difference (about 800,000 out of 1 million) unexplained. Hence, in Section \ref{sec:reach-exploratory}, we provide a more speculative discussion of the possible mechanisms behind the remainder of the difference. 

\subsection{Potential explanations}\label{sec:reach-list}

Below we list potential reasons for the differences in the number of individuals reached according to our main estimates (Table \ref{tab:headlines-vil}) and to estimates obtained through the Adoption Monitoring methods (Table \ref{tab:headlines-am}).

\begin{enumerate}[label=\quad\alph*)]
    \item the unit of reference (village versus water point); 
    \item sample selection (of water points with a higher average number of individuals per water point into the promoter survey); 
    \item promoters listing households outside of the village (the household census would generally miss these if they are located more than 200 meters outside of village boundaries); 
    \item promoters listing duplicate households (one promoter listing different members of the same household as belonging to separate households, or multiple promoters listing the same household as a user, even though this household only uses one water point); 
    \item promoters listing households that are closed (permanently moved, deceased etc.); 
    \item promoters listing households that do not use the water point (and, in particular, households that report using a water point without DSW/ILC);
    \item promoters listing households that use a water point, but not as their primary source of drinking water;
    \item promoters listing households that were missed during the household census. 
\end{enumerate}

Of these, (b), (d), (e), and (f) would imply that the promoter survey estimates overstate the number of users, whereas (c) and (h) would suggest that the household census estimates understate the number of users. The effect of (g) is more ambiguous, as there may be arguments for and against including households that use a water point infrequently in the reach estimates. In our main estimate, we define access to DSW/ILC based solely on the primary water source.

We can use the available data to explore by how much the unit of observation (a), sample selection (b), and out-of-village households (c) may be causing estimates to diverge. This is done in Section \ref{sec:reach-waterfall}. However, we do not have data to determine the effects of factors (d)-(h). Section \ref{sec:reach-exploratory} contains a more speculative discussion of these factors.

\subsection{Insights from the data}\label{sec:reach-waterfall}

Figures \ref{fig:waterfall-mw} and \ref{fig:waterfall-ug} decompose the  effects of factors (a)-(c) above on the difference between the reach estimates in Tables \ref{tab:headlines-vil} and \ref{tab:headlines-am} in Malawi and Uganda, respectively. These effects are estimated by following the same methods used to create Tables \ref{tab:headlines-vil}, \ref{tab:headlines-wp}, and \ref{tab:headlines-am}, but varying the water point sample (all water points in the water point census versus just those that we have promoter surveys for); the data sources (Household Census and Household Survey - census sample - versus promoter lists and Household Survey - promoter sample); and household sample (including versus excluding out-of-village). A more detailed explanation of the steps we followed to construct the figures is present in Appendix C.

\subsubsection{Different units of reference}

One potential reason for the difference between our primary estimates and those found by replicating Evidence Action's monitoring methodology could be that the units of reference, namely the village and the water point, are different. Comparing Tables \ref{tab:headlines-vil} and \ref{tab:headlines-wp}, which differ only in their unit of reference (they both use the Household Census and Survey for data), allows us to explore this. As shown in Figures \ref{fig:waterfall-mw} and \ref{fig:waterfall-ug}, \textbf{the estimates obtained using the water point as the unit of reference are actually lower than those obtained using the village, by about 120,000 in Malawi and 240,000 in Uganda}. Hence, the unit of reference does not help explain why estimates obtained following Evidence Action's Adoption Monitoring methods (Table \ref{tab:headlines-am}) are higher than our main estimates (Table \ref{tab:headlines-vil}).

\subsubsection{Households located outside of the village boundaries} \label{tab:diff-reach-outside}

\textcolor{red}{For the household census, enumerators were instructed to not include households located over 200 meters outside of village boundaries; however, in the promoter survey, enumerators asked promoters to list \textit{all} user households. This may have lead to divergence in estimates between the two methods.}

To understand how much the inclusion of households living outside of the villages boundaries affects the difference in reach estimates, we calculate the number of people living outside of the village boundaries according to the household census and to the promoter survey. Census estimates are obtained with the water point as the unit of reference, using the same methods as in Table \ref{tab:headlines-wp}, but restricting the sample to households living outside of the village boundaries. %We estimate that there are 72k such people in Malawi and 575k in Uganda.

Then, we estimate number of people outside of the village boundaries according to the Adoption Monitoring survey, using the same methods as in Table \ref{tab:headlines-am}. %We obtain a total of roughly 285k households living outside of the village borders and using water points with DSW/ILC in Malawi, and 307k in Uganda. 
The net effect of out-of-village households is approximated by comparing these two numbers. 

\textbf{In Malawi, out-of-village households explain about 20\% of the total difference.} Roughly 210,000 more people living outside of village boundaries are estimated to be reached under the Adoption Monitoring method compared to the census method. This suggests that a village-level census may lead us to exclude some users located in nearby villages. In Uganda, on the other hand, the census method finds more out-of-village users than the promoter list, so excluding these from both datasets actually increases the difference by around 270,000 people. Hence, \textbf{in Uganda, out-of-village households do not help explain why the promoter survey numbers are higher}. 

This finding is consistent with a comparison of the average shares of households located outside the village in the different datasets. When listing households, promoters are asked how many households in their list are located outside of the village. The average reported share is 8\% in Uganda and 9\% in Malawi. In the Household Census, the average share of households included because they were within 200m of village boundaries (as opposed to inside the village) is 3\% for Malawi and 17\% for Uganda. Hence, in Malawi, our sampling method may have led us to miss households by limiting the household census to 200m within village boundaries. \textbf{This suggests that a promoter-based protocol may be better at identifying out-of-village households in a setting where villages are further apart. }

\subsubsection{Sample selection of water points into the promoter survey}

Promoter surveys were not conducted for all of the water points with DSW or ILC in the water point census. Water points included in the promoter survey, a subset of those in the water point census, could thus systematically differ from those not included. This could occur due to 1) selection of water points into the Evidence Action promoter list (which did not contain contact details for all water points identified by IPA), or 2) selection of water points into the IPA promoter survey (since water points for which we were unable to reach a Promoter/Assistant Promoter/other person responsible for the water point were not included). If water points for which we have promoter surveys have more users than the average census water point, estimates obtained using this method would be biased upwards. 

To calculate the effect of this, we compared estimates obtained using the household census and survey taking into account all water points, with estimates obtained using the same datasets, but limiting the sample to water points included in the promoter survey. \textbf{We thus find that sample selection of water points is the single largest factor explaining the difference between our main and monitoring estimates in Uganda, but it appears less significant in Malawi.} In Uganda, it accounts for approximately 800,000 individuals, more than twice the difference between Tables \ref{tab:headlines-vil} and \ref{tab:headlines-am} (this is possible as some other factors contribute to an increased difference). In Malawi, it is responsible for about 15\% of the total difference (160,000 people). 

Water points included in the promoter survey may have more users than the average census water point for two reasons:

\begin{enumerate}[label=\quad\Alph*), start=1] 
    \item Promoter survey water points may have more households using each water point, on average.
    \item Households using promoter survey water points may have more members, on average.
\end{enumerate}

To investigate (A), we test if there is a significant difference between the average number of households per water point, estimated using the household census, between water points included in the promoter survey and those not included. We find that the average number of households per water point is larger among water points in the promoter survey across countries, sample groups, and interventions. This suggests that sample selection of water points with a larger number of households per water point, even as measured by the household census, explains part of the divergence between Tables \ref{tab:headlines-vil} and \ref{tab:headlines-am}. 

One possible explanation for this trend is that promoters that are easier to reach may also be more active in promotional activities or more consistent in refilling the dispenser, both of which might lead more households to use a water point. However, since the Evidence Action protocol is to talk to a knowledgeable community member for water points whose promoter could not be reached, the effect of sample selection is likely attenuated relative to what we see in our promoter survey.

To investigate (B), we regress household size on country, intervention type (DSW or ILC), sample group (Expansion or Footprint), and whether the household was listed by a promoter. We find that given these controls, being listed by a promoter is associated with having 0.3 more household members on average (significant at 1\%), compared to a non-promoter list mean of approximately 4.8 people per household. 

To test the impact of this on our estimates, we use the average household size among households \textit{not} listed by promoters in our Adoption Monitoring method calculations. In Malawi, this decreases the estimate of the number of people with access to DSW by about 35 thousand people, which is just 1\% of the estimate in Table 3; the ILC estimate actually increases slightly relative to Table 3. For our Uganda DSW estimates, however, using average household sizes among non-promoter listed households lowers our estimate by 550 thousand people, which means this factor could more than explain the difference between Tables 1 and 3. Hence, sample selection of larger households into promoter lists explains very little of the difference in estimates between Tables 1 and 3 in the case of Malawi, but a significant share of it for Uganda.

This suggests that in the case of Malawi, of the 160,000 people of difference that we attribute to sample selection, the majority appears to be due to (A): promoter survey water points having more users on average. In Uganda, on the other hand, well over half (approximately 550,000 out of 800,000) of the effect of sample selection appears to be due to selection of larger households into promoter lists.

\subsection{Other possible causes of divergence}\label{sec:reach-exploratory}

In Malawi, sample selection, out-of-village households, and unit of observation leave about 800,000 people of the 1-million-people difference in estimates unexplained. In Uganda, the unexplained difference is only 60,000, but the initial difference of about 300,000 households was not statistically significant to start with. In both cases, the unexplained part of the difference is due to “dataset” differences, that is, the source of the information, namely whether 1) promoter lists or census self-reports are used to estimate the number of households per water point, 2) household surveys - census sample - or household surveys - promoter sample - are used to estimate average household size among water point users.

To investigate (2), we compare average household size for households using promoter survey water points between those in the promoter sample of the household survey and those in the census sample. We find that the difference is significant only at 10\% and only in some groups, which is weak evidence supporting this factor. Thus, we focus on (1) below.  

Some reasons why promoter lists may differ from self-reports are (following the numbering from Section \ref{sec:reach-list}):

\begin{enumerate}[label=\quad\alph*), start=3]    
    \item promoters listing duplicate households;
    \item promoters listing households that are closed (members have permanently moved, are deceased etc.); 
    \item promoters listing households that do not use the water point (and, in particular, households that report using a water point without DSW/ILC);
    \item promoters listing households that use the water point, but not as their primary source of drinking water;
    \item promoters listing households that were missed during the household census. 
\end{enumerate}

The total, across both countries, of the number of households per water point according to promoters is 11,284 households. Of these, 7,625 distinct households were matched to the household census. The remaining approximately 3,700 households likely include duplicates (d), households that could not be matched due to the household being closed and thus not in the census (e), households that were missed during the census (h), or households not matched due to name-matching issues, which we observed in the field.\footnote{This includes divergent spelling, using different names, and listing different household members than the household head or the adult that the field team surveyed.} Among the matched households, our estimates may still differ between self-reports and promoter lists depending on whether households are counted as users of DSW/ILC: some may report using a water point with no DSW/ILC but be listed by a promoter (f), and others may use multiple water points, and be listed by promoters of their non-primary water point (g). We explore these different possibilities, to the extent permitted by our data, below.

\begin{itemize}
    \item \textbf{Duplicate households (d):} We are not able to test for one type of duplicates, i.e., promoters listing multiple members of one household as separate households. However, we test for the latter type, i.e., multiple promoters listing the same household as a user. When the household self-reports using more than one water point, this indicates that both of the promoters may be correct (explanation g). However, if the household reports using only 1 water point, then it is a duplicate – it should not be counted towards 2 water points. Among the 7,625 distinct households listed by promoters that we matched to the household census, we identified 499 households (around 6.5\%) that are listed by promoters of more than 1 (on average, 2) water point. Of these 499, 32\% self-report using more than one water point in the household census. In the remaining 337 cases, promoters are likely double-counting households. This is only approximately 3\% of the over 11,000 households present, in total, in the promoter lists. Hence, it appears that double-counting is present, but its magnitude is limited.
    \item \textbf{Closed households (e):} Anecdotally, while in the field, we observed promoters using water point user lists from several years prior, which led to instances of households that were closed being listed. Given our data, we are not able to estimate the magnitude of this: ideally, the field team would have been shown to every house listed by the promoter to verify the list, but this was not part of our study.
    \item \textbf{Households not using a water point with DSW/ILC (f):} There is also the possibility of promoters listing households that use a different water point, and in particular one without DSW/ILC. Indeed, among households listed by promoters, we find that 16\% in Uganda and 17\% in Malawi report using a primary water point that does not have DSW/ILC. This cannot be explained by these households using multiple water points: 12\% of households listed by promoters in Uganda and 10\% in Malawi report a primary water point without DSW/ILC and report using only one water point. This suggests that promoter lists may overstate the number of users by approximately 10\% by including households that use a water point without DSW/ILC.
    \item \textbf{Non-primary users (g):} It is also possible that some households listed by promoters are indeed occasional users, but reported a different primary water point in the household census. We are limited in our ability to quantify the magnitude of this. However, we know that in the household census, over 22\% of all households in Malawi and 24\% in Uganda report using more than one water point, and among households listed by the promoter, the shares are 24\% in both countries. As an illustrative test, we assumed that all households that report using more than one water point are using a water point with DSW/ILC. This would increase our headline estimates in Table \ref{tab:headlines-vil}  by about 400,000 people in Uganda and 200,000 in Malawi, suggesting that this factor, at most, could explain the entire difference between Tables \ref{tab:headlines-vil} and \ref{tab:headlines-am} in Uganda, but only about 20\% of the difference in Malawi.
    \item \textbf{Households missed during household census (h):} Anecdotally, while in the field, we found out about some cases of households with all members absent during the household census not being included as non-responses if the field team could not find out the household members’ names from other community members. This means that our response rates may be biased upwards, and it is possible that part of the difference between the promoter lists and self-reports stems from the household census missing some households in visited villages. Although estimates in Table \ref{tab:headlines-vil} account for non-responses by scaling the number of users in each villages proportionally to the inverse of the response rate, if response rates are overstated, then this correction does not fully eliminate the downward bias.
\end{itemize}

While we are unable to quantify the precise impact of each of these factors, it is likely that the remaining difference in estimates is a mix of our headline number under-estimating and the promoter lists over-estimating the number of users. Sampling variation is also likely to be a factor, particularly given that in Uganda, the difference between estimates in Tables 1 and 3 is not statistically significant to begin with.
