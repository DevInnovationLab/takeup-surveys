\section{Estimation methods}\label{sec:methods}

This Section describes the methods followed to calculate our two main outcomes. Program reach is estimated using the number of people whose primary source of drinking water has DSW/ILC, and Section \ref{sec:methods-people} describes how this number is calculated. Section \ref{sec:methods-cl} describes how we use the TCR detection rate from household water samples to estimate adoption rates for chlorine treatment of drinking water. Section \ref{sec:methods-mon} describes how these methods are adapted to calculate estimates that approximate those that would be obtained under Evidence Action Adoption.

\subsection{Reach estimates}\label{sec:methods-people}

The number of people reacher by Evidence Action safe water program is calculated based on the number of people living in households whose self-reported primary source of drinking water has a DSW dispenser or is connected to ILC. Standard errors are clustered by village, and sample group weights are based on the population of villages and water points in each country.

We use the R package \textit{survey} to estimate the country-level reach and the confidence interval around it.\footnote{\textcite{survey}.} The point estimates are equivalent to the procedure described below, but the package also calculates the standard errors clustered by village and corrected for sampling weights:

\begin{enumerate}
    \item Restrict the household census data to households using water points with DSW/ILC as their primary source of drinking water. 
    \item In each unit of reference (village, water point), calculate the total number of people living in these households.\footnote{Due to a survey instrument error, the question on the number of household members was skipped during the first three weeks of data collection in Uganda. In these cases, we impute it using the sample group average after restricting the sample.}
    \item Take the sample group and intervention average and multiply by the number of units of reference in this group across the country. 
\end{enumerate}

Our estimates also include the adjustments listed below. We also present sensitivity checks that change each of these at a time.

\begin{itemize}
    \item We include visited villages where no ILC water collection points were functional -- which were generally replaced for this reason, unless they had water points with DSW -- in the calculation of the average number of households using water points with ILC in our sample, assigning zero ILC users to each of them.
    \item We exclude households that use water points with DSW/ILC if the water points are located outside of village boundaries from village-level estimates. This is to avoid double counting these households when multiplying the average number of users per village by the number of villages served in the country (that is, we assume this household would have been counted in the village where the water point it uses is located).
    \item We include households living up to 200 meters outside of the village boundaries.
    \item We restrict the sample to households using water points that were confirmed to have DSW/ILC in the Evidence Action database, as opposed to relying only on field officers' reports.
    \item We account for non-responses by assuming that the share of households using a water point with DSW/ILC is the same for non-respondents as for respondents. Therefore, between Step 2 and Step 3, we divide the village-level estimate by the response rate in the same village.
    \item In Step 3, when using the water point as the unit of reference, we multiply the number of water points in the country by the functionality rate within each village group and intervention. We thus assume that the functionality rate in the country is the same as that identified in our sample.
\end{itemize}

Since our unit of sampling is the village, our main estimates use the village as the unit of reference in Step 2. We also report estimates using the water point as the unit of reference, as those may be more comparable to Evidence Action's Adoption Monitoring estimates.

\subsection{Adoption estimates}\label{sec:methods-cl}

The chlorine treatment adoption rate is estimated by calculating the share of households in the census sample of the Household Survey using a DSW/ILC water point that had chlorine detected in the water sample tested during the Household Survey.

\textit{We define chlorine detection as meeting the 0.2 mg/L threshold for Total Chlorine Residual, measured using a color wheel.}
\begin{itemize}
    \item We use the color wheel for the reasons discussed in Section \ref{sec:design-chlorine} and so our measures are more comparable with Evidence Action's.
    \item We adopt 0.2 mg/L as the threshold because this is the minimum concentration detected by the color wheel.
    \item We focus on TCR because it decays more slowly than FCR and therefore better reflects whether chlorine was applied to the water (as opposed to whether it still has any disinfectant power). It is worth noting, however, that most studies of health effects of water treatment focus on FCR measures, and therefore we also report FCR.\footnote{Table \ref{tab:thresholds} presents the criteria used by the studies included in \textcite{kremer2023water}.}
\end{itemize}  

We use the R package \textit{survey} to estimate the confidence interval clustered at the village level. Our main estimates also include the adjustments listed below, which are relaxed for sensitivity checks.

\begin{itemize}
    \item We restrict the sample to households using water points that were confirmed to have DSW/ILC in the Evidence Action database. We also show how estimates change when using the raw field officers' report of DSW/ILC presence, as well as the household's report, and when not restricting the sample at all.
    \item We restrict the sample to households that had water available in the household at the time of the survey. This is a deviation from Evidence Action's Adoption Monitoring protocol, since they assume these households do not chlorinate.
    \item We calculate the unweighted average of the indicator for chlorine detection across all households in this sample. That is, every household in the data has the same weight. We use this estimate because, as far as we understand, this is the procedure followed by Evidence Action in their Adoption Monitoring. Our sensitivity checks show how using different weights affects the estimates.
\end{itemize}

\subsection{Approximating Adoption Monitoring estimates}\label{sec:methods-mon}

Evidence Action's Adoption Monitoring estimates reach and adoption starting from a list of water point users provided by a local promoter, and sampling households from that list. As described in Sections \ref{sec:design-pm} and \ref{sec:design-mon}, we also collected data to follow the Adoption Monitoring estimation methods.

To estimate the number of people with access to DSW/ILC under this method, we 
\begin{enumerate}
    \item Group water points into 4 categories: Footprint DSW, Expansion DSW, DSW in ILC villages, and ILC water points (the latter two only apply to Malawi);
    \item Calculate the average number of households listed by promoters as using water points with DSW/ILC in the group, as reported in the Promoter Survey;
    \item Calculate the average household size among households in the promoter sample of the Household Survey that use water points in the group;
    \item Multiply these two averages to obtain the average number of households per water point in each group;
    \item Multiply the average number of households per water point by the number of water points in each group according to the Evidence Action database.
\end{enumerate}

We estimated chlorination rates by finding the unweighted average across households in the promoter sample of the indicator representing whether detected FCR and FCR concentrations were above 0.2 mg/L. Here, we immediately combined all DSW water points, as opposed to first finding chlorination rates separately for Expansion, Footprint, and DSW in ILC villages. This was because the differences in chlorination rates among these categories were found to not be statistically different. 

For chlorination rates, standard errors were estimated accounting for village-level clustering and sample group stratification. Standard errors for household counts and household size were estimated accounting for clustering at the village level and stratification by sample group. For the estimated number of people per water point, standard errors were calculated using the delta method, assuming independence between household counts and household size.\footnote{The Pearson's correlation coefficient between them is 0.022 (95\% CI = -0.044 - 0.087).} Population-level estimates were obtained by scaling water point-level estimates by the country number of water points of each category, with standard errors scaled proportionally. Country totals across categories were aggregated assuming independence, with standard errors estimated as the square root of the sum of squared standard errors.
