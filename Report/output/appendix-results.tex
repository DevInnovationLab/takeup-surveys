\textcolor{red}{\section{Bullet point summary of results}}\label{sec:appendix-results}

\subsection{Uganda}

\begin{itemize}
    \item FCR concentration of at least 0.2 ppm was observed in water samples treated with chlorine from dispensers in 66\% of the water points with DSW.
    \begin{itemize}
        \item 95.1\% of the water points with DSW were functional at the time of the water point census (Table \ref{tab:dsw-func}, column 5), compared to 80.1\% of water points without DSW or ILC. 
        \item Around 90\% of the dispensers were functional, and 70\% were dispensing chlorine (Table \ref{tab:dsw-func}, column 5). Although similar, all indicators of functionality except for percent of dispensers filled with chlorine are lower in Expansion than in Footprint villages (Table \ref{tab:dsw-func-group}).
        \item Filled dispensers were releasing the expected amount of dilute chlorine solution, at 3 mL. TCR was detected in 99.2\% of the samples of water treated by the field team at the water point with chlorine from the dispenser, and FCR was detected in 96.1\% (Table \ref{tab:dsw-cl-group}). 
    \end{itemize}    
    \item We estimate that approximately \textbf{3.8 million people}\textbf{ (95\% CI = 2.9 - 4.8 million)} use water points with dispensers as their primary source of drinking water across all the villages with Dispensers for Safe Water (DSW) (Table \ref{tab:headlines-vil}, Column 4).
    \begin{itemize}
         \item Around 2.9 million of these households live in Expansion villages. These villages are larger than Footprint villages, with an average of 204 households per village compared to 114. They are also more numerous (4,012 vs 3,173), and, on average, a larger portion of the households in these villages use water points with DSW (78\% vs 66\% in Footprint villages). On average, there are also more water points with DSW in per village in the Expansion group (4) than in the Footprint group (2).
         \item We use the village as the main unit of reference for this estimate to best reflect our sampling strategy. Since Evidence Action samples water points instead of villages in their monitoring, we also estimate reach following this method. Using the water point as the unit of reference, we estimate that 3.5 million people (95\% CI 2.9 - 4.2 million) use water points with DSW across the country (Table \ref{tab:headlines-wp}, Column 4).
         \item Using data from promoters' lists of households using water points with DSW results in a country-level estimate of 4.25 million people (95\% CI = 3.7 - 4.8 million) using water points with DSW.
    \end{itemize}
    \item 42.7\% of the households using water points with DSW included in the Household Survey reported having treated their drinking water.
    \begin{itemize}
        \item Chlorine dilute solution from dispensers was the most common treatment method among these households, with 30.1\% of households reporting adoption. The second most common method, reported by 8\% of households, was boiling. 
        \item Among households using water points without DSW/ILC, self-reports from the household census indicate that 47\% usually treat the drinking water. Fifteen percent report using some form of chlorine to do it (the same adoption rate of boiling), and 8\% report using dilute chlorine solution from the dispenser. 
    \end{itemize}
     \item Out of the 1,121 households using water points with DSW that had a sample of their drinking water tested using a color wheel, the Total Chlorine Residual (TCR) reading was at least 0.2 ppm in \textbf{30.1\% (95\% CI = 24.5\% - 35.7\%)}, and the Free Chlorine Residual (FCR) in 19.4\% (95\% CI = 14.6\% - 24.1\%) (Table \ref{tab:headlines-vil}, Columns 7 and 8).
    \begin{itemize}
         \item The differences in chlorine detection rates  between Expansion and Footprint villages are not statistically significant, thus we combine the two groups.
         \item TCR was detected in 43.8\% (95\% CI = 36.7\% - 50.8\%) of the households sampled from promoter lists, and FCR in 31.5\% (95\% CI = 25.2 - 37.6\%).
         \item The TCR detection rate among households that reported treating the water with chlorine from the dispenser was 64.0\%, compared to 15.2\% among those did not report using this method.
    \end{itemize}
\end{itemize}

\subsection{Malawi}

\subsubsection{DSW}

\begin{itemize}
    \item FCR was detected at 66\% of the water points with DSW.
    \begin{itemize}
        \item Only 2.5\% of the water points with DSW were not functional at the time of the water point census (Table \ref{tab:dsw-func}, column 5), compared to 16.6\% of the water points not served by Evidence Action. 
        \item 83.2\% of the dispensers were functional (the valve could be turned), but only 58.4\% of them were filled.
        \item Dispensers in Expansion villages had lower functionality rates (74.1\%) and fill rates (48.2\%) than those in Footprint and Expansion villages (both with approximately 90\% of dispensers functional and 65\% filled). Although large, these differences are not statistically significant.
        \item Dispensers were releasing the expected dose of dilute chlorine solution. TCR was detected in \textcolor{red}{99.5\%} of the samples of water treated by the field team at the water point with chlorine from the dispenser, and FCR was detected in 95.6\% (Table \ref{tab:dsw-cl-group}). 
    \end{itemize}    
    \item We estimate that approximately \textbf{2.1 million people (95\% CI = 1.3 - 2.8 million)} use water points with DSW as their primary source of drinking water in Malawi.
    \begin{itemize}
         \item As in Uganda, most of these households, roughly 1.75 million, live in Expansion villages. There are significantly more Expansion than Footprint villages (4,809 vs 1,161). The average number of households per village is not very different between the two sample groups, but on average 75\% of the households in Expansion villages use water points with DSW, compared to 50\% in Footprint villages.
         \item Using the water point as the unit of reference, the estimate is of 2 million people (95\% CI 1.5 - 2.5 million).
         \item Using the promoter survey data, we estimate that 3.1 million people (95\% CI 2.8 - 3.4 million) use water points with DSW.         
    \end{itemize}
    \item 32.8\% of the households using water points with DSW included in the Household Survey reported having treated their drinking water.
    \begin{itemize}
        \item Chlorine dilute solution from dispensers was the most common treatment method, with 25.5\% of household reporting adopting it. No other treatment method was used by more than 3\% of households. 
        \item Among households using water points without DSW/ILC, self-reports from the household census indicate that 52.4\% usually treat the drinking water, 34\% use chlorine to do so, and 9.6\% use the dilute chlorine solution from the dispenser.
        \item More households have TCR detected \textcolor{red}{()\%} in their drinking water than self-report using chlorine to treat it (29.2\%). This can be due to the respondent not being responsible for fetching water and therefore not being aware that it had been treated. Since very few households report boiling the water, this is not a probable reason for lower chlorine detection.
    \end{itemize}
     \item Out of the 1,279 households using water points with DSW tested using a color wheel, the TCR reading was at least 0.2 ppm in \textbf{26.6\% (95\% CI = 21.1\% - 32.1\%)}, and the FCR reading in 20.9\% (95\% CI = 16.0\% - 25.7\%).
    \begin{itemize}
     \item The difference in chlorine detection rates  between Expansion and Footprint villages is not statistically significant.
     \item TCR was detected in 30.1\% (95\% CI = 25.4\% - 36.6\%) of the households sampled from promoter lists, and FCR in 27.0\% (95\% CI = 21.6\% - 32.5\%).
     \item The TCR detection rate among households that reported treating the water with chlorine from the dispenser was 68.4\%, \textcolor{red}{compared to 15.4\% among those did not report using this method.}
    \end{itemize}
\end{itemize}

\subsubsection{ILC}

\begin{itemize}
    \item As described in Section \ref{sec:data-malawi}, non-functionality of ILC water collection points turned out to be so common that we had to adapt our village replacement strategy to not replace villages where all water collection points where out of order.
    \begin{itemize}
        \item Our data includes 9 villages, served by 19 different water points, with no functional ILC water collection points at the time of the visit. This is taken into account for reach estimates; 
        \item 6 villages were served by 4 different water points that were not connected to the ILC device when the visit happened. Since the water points were functional, they were tested for chlorine residual, and these villages are included in our analysis.
    \end{itemize}
    \item Even when water collection points were functional and the ILC device was connected, access to the treated water was often more restricted than expected.
    \begin{itemize}
        \item In 1 village, the ILC device served private households in the police compound. Although the community at large does not use the water points with ILC, the field team was able to survey the households that do, and this village is included in our analysis.
        \item \textcolor{red}{According to the reports of the village guides, another five water points with ILC devices and 26 water collection points were not used by more than one compound.}
        \item In 4 villages included in the analysis, the field team was able to conduct the chlorine residual testing, but access to ILC was more restricted than expected. In 2 of them, because fewer water points were found than included in the administrative data, and in the other two 2, because some of the water points were private.
    \end{itemize}
    \item \textcolor{red}{}
    \item \textcolor{red}{Add ILC chlorination stats here}
    \item We estimate that approximately \textbf{80,000 people}\textbf{ (95\% CI = 50,000 - 100,000)} use ILC water collection points as their primary source of drinking water.
    \begin{itemize}
         \item Using the water point as the unit of reference, we estimate that 72,000 people (95\% CI = 50,000 - 100,000) use ILC water collection points across the country.
         \item Following promoter reports, we estimate that 164,000 people (95\% CI = 135,000 - 195,000) use ILC water collection points across the country.
    \end{itemize}
     \item 389 households using ILC water collection points had a sample of their drinking water tested using a color wheel. Out of these, 28.9\% (95\% CI = 19.1\% - 38.7\%) had TCR of at least 0.2 ppm, and 13.8\% (95\% CI = 8.0\% - 19.5\%) had FCR of at least 0.2 ppm.
     \begin{itemize}
         \item The TCR detection rate is 35.9\% among households using the water collection points closest to the tank where the ILC device is installed, compared to \textbf{35.9\%} among households using other water collection points. FCR detection rates are around 10 percentage points higher among households using the water collection points closest to the ILC device, at 23.1\%.
         \item Although large, these differences are not statistically significant. This is likely due to the small number of households using the water collection points closest to the tank that were tested. Therefore, disaggregated estimates may not be representative of the country-level population.
         \item Among households sampled from those included in promoter lists, 22.3\% (95\% CI = 12.1\% - 32.5\%) had TCR detected in their drinking water, and 14.8\% (95\% CI = 6.3\%	- 23.4\%) had FCR detected.
    \end{itemize}
\end{itemize}
