\begin{table}[H]
\centering
\caption{\label{tab:dsw-cl-group}Post-treatment chlorine detection rates at water points with DSW}
\centering
\resizebox{\ifdim\width>\linewidth\linewidth\else\width\fi}{!}{
\begin{threeparttable}
\begin{tabular}[t]{cccccc}
\toprule
\multicolumn{1}{c}{ } & \multicolumn{1}{c}{(1)} & \multicolumn{1}{c}{(2)} & \multicolumn{1}{c}{(3)} & \multicolumn{1}{c}{(4)} & \multicolumn{1}{c}{(5)} \\
Sample & Filled dispenser & TCR tests & TCR ≥ 0.2 ppm & FCR tests & FCR ≥ 0.2 ppm\\
\midrule
\addlinespace[0.3em]
\multicolumn{6}{l}{\textbf{Malawi}}\\
\hspace{1em}Footprint & 34 & 32 & 96.9\% & 32 & 93.8\%\\
\hspace{1em}Expansion & 41 & 41 & 95.1\% & 41 & 95.1\%\\
\addlinespace[0.3em]
\multicolumn{6}{l}{\textbf{Uganda}}\\
\hspace{1em}ILC & 43 & 41 & 97.6\% & 41 & 97.6\%\\
\hspace{1em}Footprint & 43 & 43 & 97.7\% & 43 & 95.3\%\\
Expansion & 85 & 85 & 100\% & 85 & 96.5\%\\
\bottomrule
\end{tabular}
\begin{tablenotes}
\item \textit{Note: } 
\item At each water point with a filled dispenser, water was collected using a 20 liter jerrycan, dilute chlorine solution from the dispenser was added to the water, and the chlorine residual was measured using a color wheel 30 minutes after the water water treated.
\end{tablenotes}
\end{threeparttable}}
\end{table}
