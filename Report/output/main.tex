\documentclass[10pt]{article}
\usepackage{array}
\usepackage{tabularx}
\usepackage{rotating}   
\usepackage[multiple]{footmisc} 
\usepackage{amsmath}
\usepackage{mathrsfs}
\usepackage{amsfonts}
\usepackage{graphicx}
\usepackage{caption}   
\usepackage{subcaption}    
\usepackage{breqn}
\usepackage{makecell}
\usepackage{booktabs}  
\usepackage[toc,page]{appendix}  
\usepackage{lscape} 
\usepackage{ascmac}   
\usepackage{rotating}   
\usepackage[natbibapa]{apacite}    
\usepackage{url}
\usepackage{setspace}     
\usepackage{adjustbox}
\usepackage{pdfpages}    
\usepackage{lastpage} % Required to determine the last page number for the footer
\usepackage{enumerate}
\usepackage{amssymb}
\usepackage{wrapfig}
\usepackage{mathrsfs}
\usepackage{graphicx} % Required to insert images
\setlength\parindent{15pt} % Removes all indentation from paragraphs
\usepackage[most]{tcolorbox} % Required for boxes that split across pages
\usepackage{booktabs} % Required for better horizontal rules in tables
\usepackage{listings} % Required for insertion of code
\usepackage{etoolbox} % Required for if statements
\usepackage{mathrsfs}
\usepackage{amssymb}
\usepackage{capt-of}
\usepackage{soul}
\usepackage{pdfpages}
\usepackage{hyperref}
\usepackage{geometry} % Required for adjusting page dimensions and margins
\geometry{
	paper=letterpaper, % Change to letterpaper for US letter
	top=1.2in, % Top margin
	bottom=1.2in, % Bottom margin
	left=1in, % Left margin
	right=1in, % Right margin
	headheight=14pt, % Header height
	footskip=1.4cm, % Space from the bottom margin to the baseline of the footer
	headsep=1.2cm, % Space from the top margin to the baseline of the header
	%showframe, % Uncomment to show how the type block is set on the page
}
\usepackage[utf8]{inputenc} % Required for inputting international characters
\usepackage[T1]{fontenc} % Output font encoding for international characters
\usepackage{graphicx} % Required for inserting images
\usepackage{tabularx} 
\usepackage{caption}

\title{Independent chlorine coverage surveys\\Summary of headline results}

\begin{document}

\maketitle
\date{}
\section{Primary outcomes}

\subsection{Uganda}

\begin{itemize}
    \item We estimate that approximately \textbf{3.8 million people}\textbf{ (95\% CI = 2.9 - 4.8 million)} use water points with dispensers as their primary source of drinking water across all the villages with Dispensers for Safe Water (DSW) (Table \ref{tab:headlines-vil}, Column 4).
    \begin{itemize}
         \item Around 2.9 million of these households live in Expansion villages. These villages are larger than Footprint villages, with an average of 204 households per village compared to 114. They are also more numerous (4,012 vs 3,173), and, on average, a larger portion of the households in these villages use water points with DSW (78\% vs 66\% in Footprint villages).
         \item We use the village as the main unit of reference for this estimate to best reflect our sampling strategy. Since Evidence Action samples water points instead of villages in their monitoring, we also estimate the same number following this method. Using the water point as the unit of reference, we estimate that 3.5 million people (95\% CI 2.9 - 4.2 million) use water points with DSW across the country (Table \ref{tab:headlines-wp}, Column 4).
         \item Using data from promoters' lists of households using water points with DSW results in a country-level estimate of 4.25 million people (95\% CI = 3.7 - 4.8 million) using water points with DSW.
    \end{itemize}
     \item Out of the 1,121 households using water points with DSW that had a sample of their drinking water tested using a color wheel, the Total Chlorine Residual (TCR) reading was at least 0.2 ppm in \textbf{30.1\% (95\% CI = 24.5\% - 35.7\%)}, and the Free Chlorine Residual (FCR) in 19.4\% (95\% CI = 14.6\% - 24.1\%) (Table \ref{tab:headlines-vil}, Columns 7 and 8).
    \begin{itemize}
         \item The differences in chlorine detection rates  between Expansion and Footprint villages are not statistically significant, thus we combine the two groups.
         \item TCR was detected in 43.8\% (95\% CI = 36.7\% - 50.8\%) of the households sampled from promoter lists, and FCR in 31.5\% (95\% CI = 25.2 - 37.6\%).
    \end{itemize}
\end{itemize}

\subsection{Malawi}

\subsubsection{DSW}

\begin{itemize}
    \item We estimate that approximately \textbf{2.1 million people}\textbf{ (95\% CI = 1.3 - 2.8 million)} use water points with DSW as their primary source of drinking water in Malawi.
    \begin{itemize}
         \item As in Uganda, most of these households, roughly 1.75 million, live in Expansion villages. There are significantly more Expansion than Footprint villages (4,809 vs 1,161). The average number of households per village is not very different between the two areas, but on average 75\% of the households in Expansion villages use water points with DSW, compared to 50\% in Footprint villages.
         \item Using the water point as the unit of reference, the estimate is of 2 million people (95\% CI 1.5 - 2.5 million).
         \item Using the promoter survey data, we estimate that 3.1 million people (95\% CI 2.8 - 3.4 million) use water points with DSW.         
    \end{itemize}
     \item Out of the 1,279 households using water points with DSW tested using a color wheel, the TCR reading was at least 0.2 ppm in \textbf{26.6\% (95\% CI = 21.1\% - 32.1\%)}, and the FCR reading in 20.9\% (95\% CI = 16.0\% - 25.7\%).
      \begin{itemize}
         \item The differences in chlorine detection rates  between Expansion and Footprint villages are not statistically significant.
         \item TCR was detected in 30.1\% (95\% CI = 25.4\% - 36.6\%) of the households sampled from promoter lists, and FCR in 27.0\% (95\% CI = 21.6\% - 32.5\%).
    \end{itemize}
\end{itemize}

\subsubsection{ILC}

\begin{itemize}
    \item We estimate that approximately \textbf{80,000 people}\textbf{ (95\% CI = 50,000 - 100,000)} use ILC water collection points as their primary source of drinking water.
    \begin{itemize}
         \item Using the water point as the unit of reference, we estimate that 72,000 people (95\% CI = 50,000 - 100,000) use ILC water collection points across the country.
         \item Following promoter reports, we estimate that 164,000 people (95\% CI = 135,000 - 195,000) use ILC water collection points across the country.
    \end{itemize}
     \item 389 households using ILC water collection points had a sample of their drinking water tested using a color wheel. Out of these, 28.9\% (95\% CI = 19.1\% - 38.7\%) had TCR of at least 0.2 ppm, and 13.8\% (95\% CI = 8.0\% - 19.5\%) had FCR of at least 0.2 ppm.
     \begin{itemize}
         \item The TCR detection rate is 35.9\% among households using the water collection points closest to the tank where the ILC device is installed, compared to \textbf{35.9\%} among households using other water collection points. FCR detection rates are around 10 percentage points higher among households using the water collection points closest to the ILC device, at 23.1\%.
         \item Although large, these differences are not statistically significant. This is likely due to the small number of households using the water collection points closest to the tank that were tested. Therefore, disaggregated estimates may not be representative of the country-level population.
         \item Among households sampled from those included in promoter lists, 22.3\% (95\% CI = 12.1\% - 32.5\%) had TCR detected in their drinking water, and 14.8\% (95\% CI = 6.3\%	- 23.4\%) had FCR detected.
    \end{itemize}
\end{itemize}

\section{Methodology and sensitivity checks}

\subsection{Sample definition}

\begin{itemize}
    \item Households using water points with ILC in DSW villages are excluded from our estimates because the sample was not design to be representative of such cases.
    \item Village-level estimates of the number of households using water points with DSW or ILC restrict the sample to households using water points within the boundaries of sampled villages. 
    \begin{itemize}
        \item This is because including households using water points outside of the village would lead to double counting of households when estimating the country-level population given our village-level sample design. 
        \item Figure \ref{fig:users-vil} shows how this inclusion changes our estimates.
    \end{itemize}
    \item Households are considered to be using water points with DSW or ILC if field officer matched their self-reported primary source of drinking water to a water point with DSW or ILC. We determine whether a water point has DSW or ILC by combining field team observations and pictures, village guide reports, and administrative data. 
    \begin{itemize}
        \item Figure \ref{fig:chlorine-vil} shows that using self-reported presence of DSW or ILC at the primary water source instead does not lead to meaningful changes in the chlorine residual detection rates, and neither does using only field team reports and disregarding correction from the administrative data.
    \end{itemize}
\end{itemize}
    

\subsection{Number of people using DSW or ILC water points as their primary source of drinking water}

\begin{itemize}
     \item The number of people using water points with DSW or ILC in a village is the sum of all people living in households that use water points with DSW or ILC as their primary source of drinking water.
     \item We calculate this number by multiplying the average number of people per village (or water point, if that is the unit of reference) in each combination of country, sample group (Footprint, Expansion, ILC) and intervention (DSW or ILC) and the total  number of villages (water points) in each category according to administrative data. As discussed in Section 1.1, separation between sample groups is relevant due to the larger size of Expansion village in Uganda. 
    \item ILC water collection points have lower functionality rates than DSW water points, and in some of the villages in the ILC sample group, no ILC water collection points were functional. If there were no DSW water points in these villages, they were replaced. We incorporate the lower functionality rate among ILC water collection points into our estimates by including all villages sampled from the ILC group in the calculation, and assigning zero users to those where no households are using ILC water collection points. Figure \ref{fig:users-vil} presents the estimates obtained when functionality rates are not incorporated.
    \item Non-response rates in our household census vary greatly from village to village and were higher in Uganda. Our estimates compensate for non-responses by multiplying the total number of people using DSW or ILC include in the census by the inverse of the response rate in each village. This assumes that failure to survey a household in the household census was random, and therefore households that were not included in the household census have the same average number of members and use water points with DSW or ILC in the same proportion as the households living in the same village that were included. Figure \ref{fig:users-vil}  presents the estimates obtained when non-responses are not taken into account.
\end{itemize}

\subsection{Chlorine detection rates}

\begin{itemize}
    \item Chlorine detection rates are calculated among households using water points that were considered to have DSW or ILC when combining field team observations and pictures, village guide reports, and administrative data.
    \begin{itemize}
        \item Figure \ref{fig:chlorine-vil} shows that changing the sample inclusion criterion to use self-reported presence of DSW or ILC or field reports not validated against administrative data does not change estimates.
    \end{itemize}
    \item Chlorine detection rates in Table \ref{tab:headlines-vil} represent a simple average of tested households.
    \begin{itemize}
        \item As far as we understand, this is the most comparable number to Evidence Action estimates, which is why we chose them as our main chlorine detected rates. 
        \item These chlorine detection rates do not account for cross-village heterogeneity in the number of households using DSW water points, since 20 households were sampled in each village. 
        \item Weighing observations by the number of households using water points with DSW or ILC in their respective villages is likely a better approximation of the share of households in each country that are drinking treated water. This estimate is included in Figure \ref{fig:chlorine-vil}.
    \end{itemize}
\end{itemize}

\newpage

\textbf{Malawi} &  &  &  &  &  &  &  & \\
DSW & 79 & 5,085 & 6,607 & 2,087,999 & 76 & 1,279 & 26.6 & 20.9\\
 &  &  &  & (1,333,381, 2,842,616) &  &  & (21.1, 32.1) & (16, 25.7)\\
ILC & 30 & NA & NA & 77,024 & 29 & 389 & 28.9 & 13.8\\
 &  &  &  & (51,449, 102,598) &  &  & (19.1, 38.7) & (8, 19.5)\\
\addlinespace
\textbf{Uganda} &  &  &  &  &  &  &  & \\
DSW & 60 & 4,940 & 7,185 & 3,799,026 & 60 & 1,121 & 30.1 & 19.4\\
 &  &  &  & (2,868,202, 4,729,850) &  &  & (24.5, 35.7) & (14.6, 24.1)\\

\begin{table}[H]
\centering
\caption{\label{tab:headlines-wp}Primary outcomes estimated using the water point as the unit of reference}
\centering
\resizebox{\ifdim\width>\linewidth\linewidth\else\width\fi}{!}{
\begin{threeparttable}
\begin{tabular}[t]{lcccccccc}
\toprule
\multicolumn{1}{c}{ } & \multicolumn{1}{c}{(1)} & \multicolumn{1}{c}{(2)} & \multicolumn{1}{c}{(3)} & \multicolumn{1}{c}{(4)} & \multicolumn{1}{c}{(5)} & \multicolumn{1}{c}{(6)} & \multicolumn{1}{c}{(7)} & \multicolumn{1}{c}{(8)} \\
\multicolumn{1}{c}{ } & \multicolumn{2}{c}{Household census} & \multicolumn{2}{c}{Country-level estimates} & \multicolumn{4}{c}{Percent of color wheel readings ≥ 0.2 mg/L} \\
\cmidrule(l{3pt}r{3pt}){2-3} \cmidrule(l{3pt}r{3pt}){4-5} \cmidrule(l{3pt}r{3pt}){6-9}
Intervention & \#WP & \#HH & \#WP & \makecell[c]{People using\\DSW/ILC\\(95\% CI)} & \#WP & \#HH & \makecell[c]{TCR\\(95\% CI)} & \makecell[c]{FCR\\(95\% CI)}\\
\midrule
\addlinespace[0.3em]
\multicolumn{9}{l}{\textbf{Malawi}}\\
\hspace{1em}DSW & 194 & 5,228 & 16,198 & 1,955,667 & 176 & 1,269 & 26.4 & 20.7\\
\hspace{1em} &  &  &  & (1,377,240, 2,534,095) &  &  & (21, 31.9) & (16, 25.5)\\
\hspace{1em}ILC & 136 & 1,229 & 2,024 & 65,563 & 92 & 352 & 30.7 & 14.8\\
\hspace{1em} &  &  &  & (42,793, 88,332) &  &  & (20.7, 40.6) & (8.9, 20.7)\\
\addlinespace[0.3em]
\multicolumn{9}{l}{\textbf{Uganda}}\\
\hspace{1em}DSW & 175 & 6,045 & 17,718 & 3,532,910 & 151 & 990 & 30.4 & 19.6\\
\hspace{1em} &  &  &  & (2,799,088, 4,266,732) &  &  & (24.8, 36) & (14.9, 24.3)\\
\bottomrule
\end{tabular}
\begin{tablenotes}
\item[1] The data used for these estimates includes all water points within the boundaries of sampled villages, as well as all households identified in the household census that self-report using these water points as their primary source of drinking water (regardless of whether the households live inside village boundaries or not).
\item[2] Confidence intervals are calculated based on standard errors clustered at the village level.
\item[3] The number of people using each water point is defined as the sum of all people living in households that report using it as their primary source of drinking water. During the first two weeks of data collection, data on the number of people living in a household was not collected due to a survey coding error. This affects eight villages in Uganda. For households in these villages, we impute the number of household members as the average across all households using water points with DSW in the same country and sample group (Expansion or Footprint).
\end{tablenotes}
\end{threeparttable}}
\end{table}


\begin{table}[H]
\centering
\caption{Primary outcomes estimated using the village as the unit of reference}
\label{tab:headlines-vil-by-group}
\centering
\resizebox{\ifdim\width>\linewidth\linewidth\else\width\fi}{!}{
\begin{tabular}[t]{cccccccccc}
\toprule
\multicolumn{1}{c}{ } & \multicolumn{1}{c}{ } & \multicolumn{1}{c}{(1)} & \multicolumn{1}{c}{(2)} & \multicolumn{1}{c}{(3)} & \multicolumn{1}{c}{(4)} & \multicolumn{1}{c}{(5)} & \multicolumn{1}{c}{(6)} & \multicolumn{1}{c}{(7)} & \multicolumn{1}{c}{(8)} \\
\multicolumn{2}{c}{ } & \multicolumn{2}{c}{Household census} & \multicolumn{2}{c}{Country-level estimates} & \multicolumn{4}{c}{Percent of color wheel readings ≥ 0.2 ppm} \\
\cmidrule(l{3pt}r{3pt}){3-4} \cmidrule(l{3pt}r{3pt}){5-6} \cmidrule(l{3pt}r{3pt}){7-10}
  & Sample group & \#Vil & \#HH & \#Vil & \makecell[c]{People using\\DSW/ILC\\(95\% CI)} & \#Vil & \#HH & \makecell[c]{TCR\\(95\% CI)} & \makecell[c]{FCR\\(95\% CI)}\\
\midrule
\addlinespace[0.3em]
\multicolumn{10}{l}{\textbf{Malawi}}\\
\hspace{1em}DSW & Expansion & 28 & 2,141 & 11,292 & 1,752,945 & 28 & 520 & 27 & 21.7\\
\hspace{1em} &  &  &  &  & (1,161,309, 2,344,580) &  &  & (17.1, 36.8) & (12.1, 31.2)\\
\hspace{1em}DSW & Footprint & 27 & 1,273 & 3,844 & 242,716 & 25 & 442 & 30.1 & 21.6\\
\hspace{1em} &  &  &  &  & (175,492, 309,940) &  &  & (20.5, 39.7) & (14.9, 28.2)\\
\hspace{1em}DSW & ILC & 24 & 1,702 & 1,062 & 95,998 & 25 & 331 & 20.6 & 18.1\\
\hspace{1em} &  &  &  &  & (62,460, 129,536) &  &  & (13.8, 27.3) & (11.7, 24.5)\\
\hspace{1em}ILC & ILC & 28 & 1,191 & 2,024 & 72,327 & 27 & 368 & 30.7 & 14.8\\
\hspace{1em} &  &  &  &  & (47,054, 97,599) &  &  & (20.7, 40.6) & (8.9, 20.7)\\
\addlinespace[0.3em]
\multicolumn{10}{l}{\textbf{Uganda}}\\
\hspace{1em}DSW & Expansion & 30 & 3,467 & 12,262 & 2,895,778 & 30 & 588 & 32.7 & 21.2\\
\hspace{1em} &  &  &  &  & (2,150,157, 3,641,398) &  &  & (25.4, 40.1) & (15, 27.4)\\
\hspace{1em}DSW & Footprint & 30 & 1,479 & 5,456 & 907,708 & 30 & 547 & 28 & 18\\
\hspace{1em} &  &  &  &  & (657,373, 1,158,043) &  &  & (19.4, 36.5) & (10.8, 25.2)\\
\bottomrule
\end{tabular}}
\end{table}

\begin{table}[H]
\centering
\caption{\label{tab:headlines-wp-by-group}Primary outcomes estimated using the water point as the unit of reference}
\centering
\resizebox{\ifdim\width>\linewidth\linewidth\else\width\fi}{!}{
\begin{threeparttable}
\begin{tabular}[t]{cccccccccc}
\toprule
\multicolumn{1}{c}{ } & \multicolumn{1}{c}{ } & \multicolumn{1}{c}{(1)} & \multicolumn{1}{c}{(2)} & \multicolumn{1}{c}{(3)} & \multicolumn{1}{c}{(4)} & \multicolumn{1}{c}{(5)} & \multicolumn{1}{c}{(6)} & \multicolumn{1}{c}{(7)} & \multicolumn{1}{c}{(8)} \\
\multicolumn{2}{c}{ } & \multicolumn{2}{c}{Household census} & \multicolumn{2}{c}{Country-level estimates} & \multicolumn{4}{c}{Percent of color wheel readings ≥ 0.2 mg/L} \\
\cmidrule(l{3pt}r{3pt}){3-4} \cmidrule(l{3pt}r{3pt}){5-6} \cmidrule(l{3pt}r{3pt}){7-10}
  & Sample group & \#WP & \#HH & \#WP & \makecell[c]{People using\\DSW/ILC\\(95\% CI)} & \#WP & \#HH & \makecell[c]{TCR\\(95\% CI)} & \makecell[c]{FCR\\(95\% CI)}\\
\midrule
\addlinespace[0.3em]
\multicolumn{10}{l}{\textbf{Malawi}}\\
\hspace{1em}DSW & Footprint & 50 & 1,334 & 3,698 & 443,331 & 47 & 547 & 31.4 & 23.6\\
\hspace{1em} &  &  &  &  & (324,029, 562,632) &  &  & (23, 39.8) & (17.6, 29.6)\\
\hspace{1em}DSW & Expansion & 82 & 2,177 & 11,438 & 1,394,679 & 77 & 719 & 27.6 & 22.5\\
\hspace{1em} &  &  &  &  & (933,395, 1,855,963) &  &  & (18.4, 36.7) & (14, 31)\\
\hspace{1em}DSW & ILC & 63 & 1,717 & 1,062 & 117,658 & 59 & 460 & 23 & 20.1\\
\hspace{1em} &  &  &  &  & (75,976, 159,339) &  &  & (16.1, 29.9) & (13.3, 27)\\
\hspace{1em}ILC & ILC & 136 & 1,229 & 2,024 & 65,563 & 105 & 493 & 28.9 & 15.5\\
\hspace{1em} &  &  &  &  & (42,793, 88,332) &  &  & (19.4, 38.3) & (9, 22)\\
\addlinespace[0.3em]
\multicolumn{10}{l}{\textbf{Uganda}}\\
\hspace{1em}DSW & Footprint & 57 & 1,969 & 5,491 & 1,039,269 & 56 & 661 & 31.7 & 20.5\\
\hspace{1em} &  &  &  &  & (780,880, 1,297,657) &  &  & (23.2, 40.2) & (13.8, 27.1)\\
\hspace{1em}DSW & Expansion & 118 & 4,076 & 12,227 & 2,493,641 & 107 & 855 & 36.3 & 24.8\\
\hspace{1em} &  &  &  &  & (1,909,758, 3,077,525) &  &  & (29.5, 43.2) & (19.2, 30.4)\\
\bottomrule
\end{tabular}
\begin{tablenotes}
\item[1] The data used for these estimates includes all water points within the boundaries of sampled villages, as well as all households identified in the household census that self-report using these water points as their primary source of drinking water (regardless of whether the households live inside village boundaries or not).
\item[2] Confidence intervals are calculated based on standard errors clustered at the village level.
\item[3] The number of people using each water point is defined as the sum of all people living in households that report using it as their primary source of drinking water. During the first two weeks of data collection, data on the number of people living in a household was not collected due to a survey coding error. This affects eight villages in Uganda. For households in these villages, we impute the number of household members as the average across all households using water points with DSW in the same country and sample group (Expansion or Footprint).
\end{tablenotes}
\end{threeparttable}}
\end{table}


\begin{figure}[h!]
    \centering
    \caption{Sensitivity checks on estimates of the number of people using water points with DSW or ILC}
    \includegraphics[width=\linewidth]{output/graphs/users-vil.png}
    \label{fig:users-vil}
\end{figure}

\begin{figure}[h!]
    \centering
    \caption{Sensitivity checks on Total Chlorine Residual detection rates}
    \includegraphics[width=\linewidth]{output/graphs/chlorine-vil.png}
    \label{fig:chlorine-vil}
\end{figure}



\end{document}