\section*{Executive summary}

\subsection*{Study objectives}

\begin{itemize}
    \item The primary objective of this study is to estimate the reach and take up of Evidence Action’s Dispensers for Safe Water (DSW) program in Uganda and Malawi as well Evidence Action’s In-Line Chlorination (ILC) program in Malawi.
    \begin{itemize}
        \item We use the \textbf{total number of people living in households using water points in DSW/ILC as their primary source of drinking water} as the indicator for program reach.
        \item We use \textbf{Free Chlorine Residual (FCR) and Total Chlorine Residual (TCR) detection rates} in water samples collected at these households as the indicator for take-up of water treatment with chlorine.
    \end{itemize}
    \item The secondary objectives are to shed light on
    \begin{itemize}
        \item Whether the ILC devices and DSW dispensers are refilled properly to dose the recommended amount of chlorine.
        \item Whether Evidence Action's current monitoring protocols and the protocols designed by this study result in comparable measures of program access and adoption.
    \end{itemize}
\end{itemize}

\subsection*{Reach and adoption estimates}

\subsubsection*{Uganda}

\begin{itemize}
    \item Based on data from 60 villages, we approximate that \textbf{3.9 million (95\% CI = 2.9 - 4.8 million)} people in Uganda use water points with dispensers as their primary source of drinking water.\footnote{Table \ref{tab:headlines-vil}, column 4.}
    \item Based on the TCR tests performed with a color wheel in randomly sampled households that self-reported using water points with DSW, we estimate that \textbf{30.4\% (95\% CI = 24.8 - 36.1)} of reached households adopt chlorine as a water treatment method.\footnote{Table \ref{tab:headlines-vil}, column 7.}
    \item At the time of the surveys, chlorine was accessible at \textbf{64\%} of all the water points with DSW: 95\% of the 185 water points with DSW were functional, 70\% of dispensers in functional water points were functional and filled with chlorine, and FCR above 0.2 mg/L was detected in 96\% of the water samples treated with chlorine from filled and functional dispensers.\footnote{Tables \ref{tab:dsw-func} and \ref{tab:dsw-cl}.}
\end{itemize}

\subsubsection*{Malawi}

\begin{itemize}
    \item Our results are based on a sample of 99 hundred villages, 56 served by DSW and 43 served by ILC (24 of which are also served by DSW).
    \item We estimate that a total of \textbf{2.1 million (95\% CI = 1.4 - 2.8 million)} people are reached by DSW, with an adoption rate of \textbf{26.4\% (95\% CI = 21 - 31.9)}.\footnote{Table \ref{tab:headlines-vil}.}
    \item FCR above 0.2 mg/L was detected in\textbf{ 54\%} of the water points with DSW: 97.5\% of the 202 water points with DSW were functional, 58\% of the dispensers were functional and filled, and 96\% of the water samples treated at the water points were they were located had FCR detected.\footnote{Tables \ref{tab:dsw-func} and \ref{tab:dsw-cl}.}
    \item We estimate ILC to reach \textbf{75 thousand people (95\% CI = 49 - 102 thousand)} people, and TCR was detected in \textbf{30.7\% (CI 95\% = 24.8 - 36.1)} of the water samples tested from reached households.\footnote{Table \ref{tab:headlines-vil}.}
    \item We observe 31 ILC clusters in our data. Out of these, 90\% had an ILC device installed; 89\% these devices were functional. Around 85\% of the ILC water collection points were functional at the time of the survey. TCR was detected in water samples from 82\% of the water collection points closest to the ILC device, and FCR from 64\%. Among the other water collection points tested, only 26.5\% had FCR concentrations above 0.2 mg/L.\footnote{Tables \ref{tab:ilc-func} and \ref{tab:ilc-cl}.}
\end{itemize}

\subsection*{Comparison to Evidence Action's monitoring methods}

\begin{itemize}
    \item For comparison, we also follow a protocol similar to Evidence Action's Adoption Monitoring, obtaining user lists from water point promoters and sampling households from these. The results described below are obtained using this method.\footnote{The main deviation from Evidence Action's protocol in our surveys is that our data excludes water points whose promoter we could not identify. When Evidence Action cannot find a promoter, the interview is conducted with knowledgeable member of the community.}
\end{itemize}

\textbf{Uganda}

\begin{itemize}
    \item A total of \textbf{4 million (95\% CI = 3.5 - 4.6 million)} people are using water points with DSW,\footnote{Table \ref{tab:headlines-am}} the same result as our main estimates.
    \item The TCR detection rate is \textbf{43.4\% (95\% CI = 35.9\% - 50.9\%)}, almost 50\% higher than our main estimates.
    \begin{itemize}
        \item After considering different possible explanations,\footnote{Section \ref{sec:adoption}} a combination of sampling variation, surveyor incentives, and potential deviations from survey protocols are left as the most plausible reasons for the differences between these values and our main estimates.
    \end{itemize}
\end{itemize}

\textbf{Malawi}

\begin{itemize}
    \item \textbf{3 million (95\% CI = 2.7 - 3.4 million)} people are reached by DSW/ILC, a number that is 50\% higher than our main estimate. 
    \begin{itemize}
        \item We believe the difference in reach estimates are partly due to the over-representation of water points with more user households, which is unlikely to occur during Evidence Action's Adoption Monitoring.
        \item We also find indications that our main estimates exclude some DSW/ILC users living outside of the boundaries of sampled villages, and some households are double-counted in promoter lists despite self-reporting only using one water point.
    \end{itemize}
    \item \textbf{28.5\% (95\% CI = 23\%- 34.1\%)} of the households reached by DSW are estimated to adopt chlorine as a water treatment method, a very similar estimate to our main results. 
    \item\textbf{15.4\% (95\% CI = 6.6\% - 24.2\%)} of the households reached by ILC have TCR detected in their water, which is approximately half of our main estimate for this group.
    \begin{itemize}
        \item The number of users listed by promoters is on average 3.5 times the number of self-reported users for the same water points, and households using water points with ILC are more likely to rely on multiple sources of drinking water, which may be connected to the lower functionality rate among ILC water points.
        \item Among the households that self-report using water points with ILC, those included in the promoter list are on average 54\% less likely to have chlorine detected.
    \end{itemize}
\end{itemize}
