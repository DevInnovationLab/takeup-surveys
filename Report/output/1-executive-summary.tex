\section*{Executive summary}

\subsection*{Study objectives}

\begin{itemize}
    \item The primary objective of this study is to estimate the reach and take up of Evidence Action’s Dispensers for Safe Water (DSW) program in Uganda and Malawi as well Evidence Action’s In-Line Chlorination (ILC) program in Malawi.
    \begin{itemize}
        \item We use the total number of people living in households using water points in DSW/ILC as their primary source of drinking water as the indicator for program reach.
        \item We use Free Chlorine Residual (FCR) and Total Chlorine Residual (TCR) detection rates among these households as the indicator for take-up of water treatment with chlorine.
    \end{itemize}
    \item The secondary objectives are
    \begin{itemize}
        \item Whether the ILC devices and DSW dispensers are refilled properly to dose the recommended amount of chlorine.
        \item Whether Evidence Action's current monitoring protocols and the protocols designed by this study result in comparable measures of program access and adoption.
    \end{itemize}
\end{itemize}

\subsection*{Reach and adoption estimates}

\subsubsection*{Uganda}

\begin{itemize}
    \item Based on data from 60 villages, we approximate that \textbf{3.8 million (95\% CI = 2.9 - 4.7 million)} people in Uganda use water points with dispensers as their primary source of drinking water.
    \item Based on the TCR tests performed with a color wheel in randomly sampled households that self-reported using water points with DSW, we estimate that \textbf{30.1\% (95\% CI = 24.5 - 35.7)} of reached households adopt chlorine as a water treatment method.
    \item Across all water points with DSW identified, 95\% were functional at the time of the spotcheck, and 92\% had a dispenser with all parts present. Roughly 70\% of dispensers were filled with chlorine,\footnote{We consider a dispenser to be filled if chlorine is released when the valve is turned.} and they released the expected amount of chlorine, at 3 mL. FCR above 0.2 mg/L was detected in water samples treated with chlorine from the dispenser in 66\% of water points.
    \item \textcolor{red}{find the wording about fill and preferences}
\end{itemize}

\subsubsection*{Malawi}

\begin{itemize}
    \item Our results are based on a sample of one hundred villages, 56 served by DSW and 44 served by ILC (many of which are also served by DSW).
    \item We estimate that a total of \textbf{2.1 million (95\% CI = 1.3 - 2.8 million)} people are reached by DSW, with an adoption rate of \textbf{26.6\% (95\% CI = 21.1 - 32)}
    \item We estimate ILC to reach \textbf{77,024 (95\% CI = 51,449 - 102,598)} people, and TCR was detected in 28.9\% (CI 95\% = 19.1 - 38.7) of the water samples tested from reached households.
    \item \textcolor{red}{summary of DSW and ILC functionality} 
\end{itemize}

\subsection*{Comparison to Evidence Action's monitoring methods}

\begin{itemize}
    \item \textcolor{red}{how our methods are difference from EA's}
    \item In Uganda, following methods similar to Evidence Action's, we estimate that 
    \begin{itemize}
        \item A total of \textbf{4 million \textcolor{red}{CI}} people are using water points with DSW in Uganda. Although this number is larger than our main estimate by \textcolor{red}{half a million people}, the difference between the two results is not statistically significant and can largely be explained by differences in protocols.
        \item \textcolor{red}{Luiza to summarize Uganda chlorination rate}
    \end{itemize}
    \item In Malawi, we find that 
    \begin{itemize}
        \item 3 million people are using these water points (one million more than our main estimate). Although we are unable to test for this, we believe that this is because...
        \item \textcolor{red}{Luiza to summarize chlorination rate}
    \end{itemize}
\end{itemize}