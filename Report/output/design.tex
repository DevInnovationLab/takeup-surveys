\section{Survey design}

\subsection{Sampling strategy}

The main objective of this survey was to identify households whose primary source of drinking water had a DSW dispenser or was connected to ILC, and estimate what share of these households used chlorine to treat their drinking water. 

Therefore, we first needed to identify the households that used these water points. Evidence Action does that by asking a knowledgeable person which households are those. However, third-party reports are noisy, and the reporter may have incentives to choose who to include in this list. This survey was meant to use a more reliable method, even if more costly. As such, we chose to use self-reports on water point usage, under the assumption that households are the most knowledgeable party on their own activities. 

To do so, we had to first establish a spatial boundary for the data collection. One option would be to select a radius around water points with DSW and ILC and survey the households within this radius. However, we had little data available to inform the choice of such a radius, and the implementation of this sampling strategy in the field would be challenging. Instead, we use villages as the sampling unit because we expect them to have more explicit and well-known boundaries that were relatively organically formed around a contiguous area of easy access for those within it. 

We also wanted DSW estimates to be representative of both Footprint and Expansion areas, and in Malawi we wanted them to be representative of both DSW and ILC. To this end, we aimed to sample 30 villages where DSW had been active since before 2021 (called the Footprint sample group from now on), 30 villages where DSW activities started in 2021 or later (called the Expansion sample group from now on), and 40 villages served by ILC (called the ILC sample group from now on).\footnote{If a village was served by ILC, it was included in the ILC sample group, regardless of whether it was also a DSW Footprint or Expansion village.} 

Sampling villages from the DSW sample groups was simple, since they could be treated as independent units, but sampling villages in the ILC sample group was more complicated, since multiple villages may be part of the same pipeline. That is, the water may be pumped from one borehole, stored in a tank with an ILC device, and then distributed to water collection points across multiple villages through pipes. This means that there is a higher correlation between chlorine doses identified in villages served by the same ILC device. When monitoring their water points, Evidence Action estimates the number of people using all of the water collection points connected to the water point with an ILC device. In order to replicate more closely this effort, if a sampled village was part of an ILC cluster(that is, a group of villages with water collection points served by the same ILC device), we considered the entire cluster to be sampled. For each sampled cluster, up to five villages were included in our sample. If a cluster covered more than five villages, the village where the ILC device was located was always included, and four other villages from the cluster were randomly sampled. As a result, a total of 41 villages split into 31 clusters were sampled. 

Large ILC clusters are overrepresented in our sample. In Malawi, 95\% of the ILC clusters include up to 5 villages. The 13 large clusters out of the 285 existing ones include up to 73 villages each. This means that 244 out of 401 villages served by ILC are part of large clusters. Since we sampled at the village, not the cluster level, villages in large clusters are overrepresented. However, we have found that these villages are not different from villages that are not part of clusters or those that are part of small clusters with regard to the data available. Therefore, in the analysis, we have treated villages in small and large clusters the same way as we treated villages that are not part of any clusters.

No statistically significant differences are found in the number of DSW water points between sampled and non-sampled villages or visited and non-visited villages, supporting the representativeness of the study sample.

%-------------------------------------------------------------------------------

\subsection{Data collection instruments and protocols}\label{sec:design-instruments}

The data collection included five activities, listed below by order of chornolory of implementation. The fieldwork was led by IPA, with DIL staff being present for piloting in Uganda and during the first few weeks of data collection in Malawi. DIL also conducted the data quality monitoring

\subsubsection{Village mapping}

Upon arrival at a village, field officers asked the village chairperson to appoint a knowledgeable person to guide them around the village. During this guided walk, they mapped the village boundaries and the water points in the village. 

GPS coordinates were collected during this exercise, in an attempt to obtain a clear map of the village boundaries. However, this data turned out not to be useful, given that teams were mostly walking through roads, with guides pointing to the boundaries rather than walking around them, since they were often hard to reach. 

\subsubsection{Water point census}

During the guided walk, field officers recorded the GPS coordinates of the water points identified. If a water point was used by more than one compound to collect drinking water, additional data was collected, including the type of water source, if it was functional, and whether it had a DSW dispenser or ILC. 

Note that the inclusion criterion here is not exactly the same that Evidence Action uses to determine whether a water point is eligible to receive an intervention, which relies on \textcolor{red}{at least ten \textit{households} using a water point to collect drinking water}. This may have led to the exclusion of some Evidence Action water points from the water point census. A map with the location of all water points served DSW or ILC was provided to the field teams to minimize this issue, but during data quality checks we noticed that some teams were not using these maps consistently. 

A water point was classified as having a DSW dispenser if at least one of the three dispenser components was present during the visit: the casing, the PVC pole, and the chlorine tank. If a water point had a DSW dispenser, we collected data on dispenser functionality and measured chlorine residuals. If a water point with a DSW dispenser was functional, field officers may have tested a sample for chlorine residual before adding chlorine to it, depending on the reagent availability at the time, to better understand the sources of noise and potential false positive in chlorine measurements.\footnote{Testing of untreated water was introduced a few weeks into the data collection. Some teams in Malawi ran out of colorimeter reagents before the end of the data collection. These teams were oriented to not test the untreated water with the colorimeter to spare resources.} If the dispenser was functional and filled, field officers filled a jerrycan with water, turned the valve once, and waited for 30 minutes before testing the water for chlorine residual again. They also measured the amount of chlorine dispensed. 

There are two different types of water points with ILC:
\begin{itemize}
    \item Taps from which the end user collects water, referred to as a \textbf{water point connected to ILC} or \textbf{ILC water collection point} from now on.
    \item Water points that feed the tank to which the ILC device is connected and from where water is distributed to the taps, referred to as \textbf{water points with an ILC device} or \textbf{ILC water point} from now on.
\end{itemize}
If multiple water collection points are connected to the same water point with an ILC device, the set of all the water points connected to each other through a pipe and sharing an ILC device is called a \textbf{cluster}.

ILC devices were not always immediately visible or accessible, and field teams often relied on their guide to determine whether an ILC was present in a water point. The tanks where ILC devices are installed are typically elevated and sometimes covered, meaning the ILC device may not be visible from the ground. For safety reasons, field officers were told not to climb up to the tanks. 

A similar challenge arises with water collection points connected to ILC: it is not possible to observe if a tap is connected to a water point with ILC. Therefore, the field team had to rely on a knowledgeable local source to determine which water collection points were connected or not. This person was usually the village chairperson or a guide appointed by the chairperson to take the team to the water points in the village. At the water points, teams often talked to the ILC operators as well. Whether or not a tap is connected to ILC is the most noisy water point-level treatment variable in our data.

If a water point with an ILC device was functional and field officers could reach the ILC device without climbing the tank, they conducted a spotcheck to see if the ILC device was functional and filled with tablets. We expected the chlorine residual levels to be similar across water points supplied by the same tank. Therefore, if an ILC water point was connected to more than one functional water collection point in the village, only the one that was the closest to the ILC device and the one that was the farthest were tested, meaning we should have two water points tested per cluster per village. 

Teams were asked to take pictures of the water points, the dispensers and the ILC devices. A water collection point's connection to the ILC device cannot be directly observed, field teams relied on local guides or ILC operators to determine whether a tap water served by ILC. 

When field reports of whether a water source was served by Evidence Action did not match the administrative data from Evidence Action, DIL researchers checked the pictures taken by field teams at the water source. For ILC water collection points, field teams were asked to call their local guides or the surveyed promoters to confirm whether a water tap was connected to a tank with ILC in case of inconsistent reports.

\subsubsection{Household census}

All households identified within the boundaries of sampled villages were invited to participate in the household census. Field teams were also instructed to include all the households located within 200 meters of the village boundaries in the census, since these households might also use water points located within the village boundaries. However, the implementation of such strategy depends on the field team’s estimation of this distance, which we expect to be noisy.

If a household could not be reached in the first attempt, or an eligible respondent (a household member above 18 years of age who is knowledgeable about water, typically a woman) was not present, field officers made two more attempts, over two consecutive days, to survey the household. Households that could not be surveyed after these attempts, were not present in the village when the household census was happening, or declined to participate, are considered non-responses. 

\textbf{During the census, field officers asked households what their primary source of drinking water was, and tried to match that source to the water points identified in the water point census. }They also inquired about the number of water sources used, basic household demographics, and household composition. Finally, if a household's primary water source was not served by Evidence Action, they also asked about the household's water treatment practices. Households whose primary source of drinking water was served by Evidence Action were not asked about this to avoid priming.

Field officers used water source names and type, as well as landmarks around the water point, to match the household’s report of primary source of drinking water to the water points identified in the census. They also indicated how confident they felt that the match was correct. If the household was using a private water point, rainwater, or a water point outside of the village boundaries, it would likely not be matched. If the household reported using a communal water point inside the village, but that water point wasn’t mapped, field teams were asked to map that water point after the census was concluded to determine whether a dispenser or ILC device was present, or if the water point was connected to an ILC device.

Our main estimates of the number of people served by Evidence Action are based on data from this survey.

\subsubsection{Household survey (census sample)}

At each sampled village, we randomly sampled 20 households among those matched to a water point served by Evidence Action to complete an additional survey. This more detailed household survey focused mainly on water treatment practices and included testing a sample of drinking water for chlorine residual.

The main objective of this survey was to estimate household-level chlorination rates. Each household was asked for a cup of drinking water stored at the household. Field officers tested the water samples for both TCR and FCR, and asked households whether they had done anything to that water to make it safeer to drink.\footnote{At the start of data collection, households were surveyed even if they didn’t have any water available for testing. When we noticed that this happened more frequently than anticipated, we a new household to the sample whenever this was observed. Because the data collection started first in Uganda, the proportion of households with water available is smaller for that country.}

Each team of field officers included five people, and each one of them had a color wheel to conduct household-level water tests. Colorimeters were also available, but in a smaller number, since they are considerably more expensive. As a result, during each day of the household survey, the colorimeters rotated among team members, who, on the days when they had the two instruments, used both of them to test the same water sample. In Malawi, each team had two colorimeters. In Uganda, the second batch of colorimeters arrived in the country after teams had already left the capital. As a result, each team carried a single colorimeter to sampled villages.\footnote{In Malawi, a few enumerators initially misunderstood when to use the colorimeter or the color wheel, and did not use a color wheel when they had a colorimeter with them. As a result, some households have missing color wheel observations. We asked these enumerators to test one extra household per village with both the color wheel and the colorimeter to make up for the lost sample to be used in the comparison of instruments.}

Our main estimates of chlorine adoption rate among households served by Evidence Action are based on data from this survey. The survey also included a module on knowledge about water treatment and perception of ILC or DSW. However, a coding issue introduced after the pilot disabled this module, very few observations were collected, preventing this analysis from being completed.

\subsubsection{Promoter survey}

Promoters were identified through a list provided by Evidence Action (thereafter referred to as the Evidence Action list) which included the names and contacts of main and assistant promoters in each of the sampled villages. Field officers reached out to the main promoter to schedule a visit, and to the assistant promoter if the main promoter was not available. In some cases, when both the main and the assistant promoters were not included in the Evidence Action list, were unkowen in the village, or had moved out of the village, field officers inquired about the person responsible for the water point, and surveyed the people they were able to identify through this method.

The promoter was only contacted by the IPA team after the previously listed activities were complete. This usually happened around the fourth day of activities in a village. However, by the time promoters were contacted, they were often already aware of the ongoing surveys. In fact, a few promoters introduced themselves to the field team during the water point census and asked to participate in the activities. In such cases, teams were instructed to thank the promoters for their availability and let them know that they would be contacted when needed.

The promoter survey included a series of questions about the promoter activities and, most importantly, asked for a list of the users of a water point.\footnote{Note that the promoter is unlikely to know whether a household uses more than one water point or which one is the primary source of drinking water, so this exercise is not expected to yield exactly the same list of users as the household census.} If a water point was not functional at the time of the survey, the promoters were not asked for a list of users. The names in this list were than matched to the names of households in the household census. This was a time-consuming process, and often times required several clarifications about the use of first names, last names or common names, as well as of different members of a household. Information about the number of households living outside of the village boundaries was collected to help understand mismatches.

This survey is used to obtain an estimate of the number of people served by Evidence Action that is equivalent to Evidence Action's regular monitoring estimates. One key difference between our protocol and that of Evidence Action is treatment of water points whose promoter we could not reach. When they cannot reach either the promoter or assistant promoter, Evidence Action interviews the most knowledgeable other stakeholder in the community. In our protocol, the water point would not be included in the promoter survey if neither the promoter nor the assistant promoter could be reached, unless it was the case that they were both unknown or had moved out and someone else was actually responsible for the water point.

\subsubsection{Household survey (promoter sample)}

In addition to the 20 households sampled from the household census to complete the more detailed household survey, we also randomly sampled four households from each of the lists of users obtained from the promoters to complete the same survey. In other words, this is a household survey identical in design to the census-based household survey, including testing water samples for chlorine residual, but with the sample drawn from the promoter survey. This data is used to obtain an estimate of chlorine adoption rate among households served by Evidence Action that is equivalent to Evidence Action's regular monitoring estimates. 


\subsubsection{\textcolor{red}{Chlorine testing}}\label{sec:design-chlorine}

%-------------------------------------------------------------------------------