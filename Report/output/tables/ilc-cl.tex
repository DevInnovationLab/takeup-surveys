\begin{table}[H]
\centering
\caption{\label{tab:ilc-cl}Chlorine detection rates at water collection points connected to ILC}
\centering
\resizebox{\ifdim\width>\linewidth\linewidth\else\width\fi}{!}{
\begin{threeparttable}
\begin{tabular}[t]{lccc}
\toprule
\multicolumn{1}{c}{ } & \multicolumn{1}{c}{(1)} & \multicolumn{1}{c}{(2)} & \multicolumn{1}{c}{(3)} \\
  & N & TCR ≥ 0.2 mg/L & FCR ≥ 0.2 mg/L\\
\midrule
All water collection points & 60 & 45\% & 33.3\%\\
Closest water collection point & 11 & 81.8\% & 63.6\%\\
Other water collection point & 49 & 36.7\% & 26.5\%\\
\bottomrule
\end{tabular}
\begin{tablenotes}
\footnotesize
\item \textit{Note: } 
\item In each village, the closest and the furthest water collection point from the ILC device are tested. The sample in the first line includes all water collection points tested. The sample in the second line includes only the water collection points that were the closest to the tank in their cluster. The third line excludes the water collection points that were the closest to the tank in their cluster. Large clusters cover multiple villages. Therefore, there may be multiple water collection points in a single cluster included in lines one and three.
\end{tablenotes}
\end{threeparttable}}
\end{table}
