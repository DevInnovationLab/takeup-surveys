\section{Introduction}

Approximately \textbf{XX} of the global population lacks access to safely managed drinking water services, with contaminated water contributing to over \textbf{500,000 deaths annually}, primarily from diarrheal diseases that disproportionately affect children under five years of age (). The burden of waterborne illness concentrates in Sub-Saharan Africa and rural communities, where geographic dispersion increases infrastructure costs, absence of electrical grid access limits treatment options, and institutional capacity constraints hinder maintenance of centralized water systems \textbf{cite something for this}. Chlorine treatment represents a cost-effective intervention for household water disinfection, as chlorine compounds inactivate bacterial and viral pathogens through oxidation processes, require minimal training for application, and remain stable during storage and distribution \textbf{ictation}. The low cost per liter treated, combined with the availability of chlorine products in tablet or liquid form, positions chlorine disinfection as a scalable solution for populations without access to centralized treatment facilities.

\textcolor{red}{\textbf{need one para here explaining why safe water is important. can't be bothered to write it.} The World Health Organization estimates that in 2015, over two billion people consumed drinking water contaminated with feces and that approximately half a million people died from diarrheal disease associated with fecal contamination of water (WHO 2019). More recent estimates suggest that over 1.5 million people continue to die from diarrheal diseases annually (2). Fecal contamination is a major transmission route for infectious agents causing diarrhea, and the problem is intensifying as climate change and aquifer depletion threaten existing sources of clean water (3). Chlorination has been found to be effective in reducing the concentration of diarrheal pathogens like E.coli in controlled laboratory settings (4–7) and in reducing caregiver-reported diarrhea (8, 9). Dilute chlorine solution effectively kills many diarrheal pathogens while remaining safe and inexpensive (CDC 2018). Despite the proven effectiveness and low cost of chlorination, 71\% of the population in low-income countries and 40\% in lower-middle-income countries do not have access to safely managed drinking water facilities (10), leaving a substantial gap between available technology and implementation on the ground.}

This study focuses on two chlorine delivery methods deployed by the non-profit Evidence Action in Uganda and Malawi. The Dispenser for Safe Water program (DSW) installs chlorine dispensers at communal water sources, which allow residents to treat their water at the point of collection. In villages were piped water infrastructure is already available, Evidence Action uses In-Line Chlorination technology (ILC) to treat water before it reaches households. DSW was first implemented in Uganda in \textbf{XXX}, and in Malawi in \textbf{XXX}. In \textbf{XXX}, Evidence Action expanded its activities, covering \textbf{XXX} villages with DSW and \textbf{XXX} with ILC between \textbf{X} date and \textbf{X} date.\footnote{Numbers and dates are based on administrative data provided by Evidence Action.} 

As part of its activities, Evidence Action monitors both the reach and adoption of its programs, where reach means the number of people that receive access to safe water through the interventions and adoption means the use of chlorine to treat drinking water, or the share of the reached population that consumes water treated with chlorine. To this end, it conducts quarterly Adoption Monitoring activities that involve identifying the households using water points with DSW/ILC and testing a sample of their drinking water for Chlorine Residual.

This document reports the results of an independent exercise to estimate the same outcomes -- reach and adoption of DSW and ILC. Since it is a one-time data collection, it can be more time-consuming and expensive than regular Monitoring, Learning, and Evaluation activities. Our primary goals are:

\begin{enumerate}
    \item The total number of people using water points with DSW in Uganda, and the total number of people using water points with DSW and ILC in Malawi.
    \item The adoption rate in Uganda and Malawi, as measured by free and total chlorine residual detection using colorimeters and color wheels.
    \item How these estimates vary across villages that are part of the original DSW Footprint (where activities started in 2012 in Uganda and 2013 in Malawi), those that are part of the post-2020 DSW Expansion, and those where In-line Chlorination is offered instead of or in tandem with DSW.
\end{enumerate}

To answer these questions, we conducted a series of surveys in the two countries. We randomly sampled 60 villages served by DSW in Uganda and 100 villages served by DSW and/or ILC in Malawi; collected data on all of the water points in sampled villages; mapped all of the households living in these villages to the water points they use collect drinking water; and collected data on water treatment behavior among randomly sampled households using water points with DSW/ILC.   

Table 1 summarizes our findings. We estimate that around \textit{3.5 million people in Uganda and 2 million people in Malawi use water points served by Evidence Action programs as their primary source of drinking water.} We arrive at this estimate by calculating the average number of people living in households using Evidence Action water points in each group of villages, then multiplying this number by the total number of villages in the country belonging to each group according to Evidence Action. Country-level estimates are a weighted sum of the group-level estimates. 

\textit{We also estimate that the adoption of chlorine for treating drinking water ranges between 22.8\% and 32\% of the households using water points with DSW/ILC}, depending on the village group. This is measured by the percent of tested households where the Total Chlorine Residual detected using a color wheel is above 0.2 mg/L. We use Total Chlorine Residual as our preferred estimate because it takes longer to evaporate than the other commonly used indicator, Free Chlorine Residual. TCR indicates whether the water was ever treated with chlorine, even if some of the protective effect has faded, while FCR measures whether there is still chlorine available to combat new contaminants that may come into contact with the water. 

Apart from these main outcomes of interest, we also seek to better understand two secondary questions:
\begin{enumerate}
    \item Are in-line chlorination devices and DSW dispensers refilled properly to dose the recommended amount of chlorine?
    \item Do Evidence Action's current monitoring protocols and the protocols designed by this study result in comparable measures of program access and adoption?
\end{enumerate}

To answer the first question, we conducted visited at all of the water points served by Evidence Action identified in the sampled villages, observing whether they were functional, and dispensing chlorine, as well as measuring FCR at the water point.\footnote{We use FCR instead or TCR when looking at dosing at water point because households will not consume the water immediately, and therefore the presence of free chlorine is important to reduce contamination between fetching and consuming the water.} Over 95\% of the water points with DSW dispensers found were functional, and 60\%-70\% of all dispensers were functional and filled with chlorine. Over 96\% of the water samples treated with dilute chlorine solution from the dispensers had over 0.2 mg/L of FCR. Functionality rate at water points with ILC devices is lower: around 85\% of the water points with ILC identified in Malawi were found to be functional at the time of the visit. \textcolor{red}{Around 30\% of the water samples from water collection points connected to ILC tested had FCR readings above 0.2mg/L.}

To answer the question about Evidence Action's monitoring estimates, we adapted their Adoption Monitoring protocols, surveying the promoters responsible for the water points at the villages to list all the households using each water point, and then randomly sampling four households from each of these lists to have a sample of their drinking water tested for chlorine residual. The estimates obtained using this method, \textcolor{red}{of around four million people using EA water points in Uganda and three million in Malawi, are 1.2 and 1.45 times as large as the estimates obtained when the village is the main unit of observation. The observed chlorination rates are also higher than those obtained using the protocol we designed.}

Sections \ref{sec:reach} and \ref{sec:adoption} discuss potential explanations for these inconsistencies. \textcolor{red}{Briefly, in the promoter survey, some households are double-counted despite self-reporting only using one water point. Promoters may also list more than one person from the same household, although we are unable to verify this with the available data. On the other hand, the household census may fail to capture some user households located outside of the village (although we included households living within 200 m of sampled village boundaries in the household census to reduce this issue), as well as those subject to non-response. We believe that the higher chlorination rates result from sample-selection at the water point and potentially the household level. The larger number of households per water point maybe be reconcilable with higher chlorination Trates if we consider that promoter lists include non-identified users and some duplicates.}

The rest of the report is organized as follows. Section \ref{sec:design} describes the survey design, the survey instruments and summarizes the key Standard Operating Procedures. Section \ref{sec:data} describes our sample, discusses observations from the field, and presents descriptive statistics. Section \ref{sec:methods} explains the estimation methods and explain the research decisions made. Section \ref{sec:results} presents and discusses our findings. Sections \ref{sec:reach} and \ref{sec:adoption} include explorations of potential reasons for the differences between our main results and those find when approximating the Evidence Action's Adoption Monitoring protocols. Section \ref{sec:conclusion} concludes.