\section{Appendix to Section \ref{sec:reach}}

This Appendix details the steps used to estimate the relative importance of various factors in explaining the difference between the estimates based on the Census (Table \ref{tab:headlines-vil}) and Promoter Samples (Table \ref{tab:headlines-am}).

\begin{enumerate}
   
    \item \textbf{Exclude census out-of-village households:} We first excluded out-of-village households from the Household Census estimates, and proceeded to compare these with estimates excluding out-of-village households in the next several steps. This allowed us to isolate this effect from other factors. The village-level estimates of reach excluding out-of-village households are reported in  Figure \ref{fig:users-vil}.
    
    \item \textbf{Unit of observation:} To estimate the difference that the unit of analysis (village versus water point) makes, change the unit of reference from the village to the water point, while still using data from the Household Census and Household Survey and excluding out-of-village households.

    \item \textbf{Sample selection:} We then estimated the effect of sample selection by changing the sample from all water points in the census to only promoter survey water points, while still using the same data sources - the household census and survey.

    \item \textbf{Dataset:} Dataset differences stem from whether 1) the promoter list or self-reports are used to estimate user households, 2) the Household Survey or the Monitoring Survey are used to estimate average household size. By comparing estimates obtained using the household census and survey with estimates for the same water points obtained using the promoter and monitoring survey , we isolate the effect of the dataset. This is the "unexplained" part of the difference in our main estimates in Tables 1 and 3 in the sense that it cannot be attributed to any of the reasons (a)-(g) listed in Section 7.1.

    \item \textbf{Promoter survey out-of-village households:} We compared estimates from Table \ref{tab:headlines-am} to estimates produced in an identical manner but excluding out-of-village households to approximate the number of out-of-village households in the monitoring approach estimates. 
    
\end{enumerate}

