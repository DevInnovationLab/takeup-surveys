\section{Final remarks}\label{sec:conclusion}

This study collected data to estimate the number of people using water points included in Evidence Action’s Dispensers for Safe Water (DSW) and In-line Chlorination (ILC) programs, as well as the percent of people reached who are consuming water treated with chlorine. Using village-level censuses, household surveys, and water point inspections, we estimate that approximately four million people in Uganda and two million people in Malawi rely on water points with DSW/ILC as their primary source of drinking water.

Adoption rates, as measured by the presence of Total Chlorine Residual (TCR) in household water samples, range between 27\% and 33\% across village groups in both countries. While this represents a meaningful level of protection, it also underscores the challenge of ensuring consistent take-up and sustained use of trerated water at the household level. The data indicates that these challenges are different in DSW and ILC systems.

When DSW dispensers are functional and filled, they are generally dosing at the recommended levels, suggesting that gaps in adoption of this program are related to the reliability of refilling among promoters and to household preferences and behaviors. At the time of the surveys, chlorine was available at 64\% (95\% $\times$ 70\% $\times$ 96\%) of the water points with DSW.

In ILC systems, the water point functionality rate (85\%) is low when compared to water points with DSW (98\%). This may stem from the greater complexity of ILC infrastructure and devices. The fact that FCR detection rates are considerably lower than TCR detection rates indicates that although chlorine was applied to most of the clusters tested, the dosing  was not sufficient for FCR to be detectable given the chlorine demand in the water and tank. The different FCR detection rates in the water collection points closer to the tank with the ILC device(67\%) and further from it (26.5\%), point to additional chlorine demand in the pipeline or to chlorine decay as the water is distributed through the pipes, so not all the users of a cluster drink similarly safe water. 

In spite of the challenges with ILC operation, from a policy perfective, maintaining of a limited number of water points may be simpler problem than than promoting behavioral change among promoters (to refill dispensers) and users (to add chlorine to the water), meaning that ILC may be easier way to increase access to safe water. 

Using either the water point or the village as the unit of reference to extrapolate estimates to the country level resulted in similar estimates. This finding suggests that a water point-based data collection, which we expect to be easier and cheaper to carry out than a census-based one, may be a viable solution to measuring reach and adoption.  The challenge remains, however, of also identifying the households using the water sources with DSW or ILC with a less resource-consuming process than a census. The data from this study may help determine an appropriate catchment radius to be used in such a model.

Following Evidence Action’s Adoption Monitoring methods, on the other hand, yielded higher estimates of the number of people using water points with DSW and ILC in Malawi than our main results. It also resulted in higher Chlorine Residual detection rates in Uganda and lower rates in villages served by ILC. Chlorine detection rates from Evidence Action's DSW Q3 Adoption Monitoring are also higher than those obtained using a village-level reference. 

After this analysis, two main hypothesis to explain these contrasting estimates remain, neither of which we can be conclusively tested with the available data. That first is that differences in reach estimates are likely due to an overestimation of the number of DSW/ILC users by promoters. The second, that differences in adoption estimates are likely due to surveyor bias.

We find no evidence to refute the idea that a methodology that samples water points, relies on self-reported use of water points with DSW/ILC, and uses independent surveyors to test water samples should lead to consistent reach and adoption estimates.

Two other questions raised at the onset of this study were not discussed or discussed to a lesser extent. The first one relates to the use of different instruments to test water samples for chlorine residual. Our initial exploration indicates colorimeters are less reliable when used at scale in the field. However, a more in-depth analysis of the data is necessary to determine how these
findings on field protocol recommendations.

The other question raised in this study’s concept note regarded the  factors that influence the choice of water source and the decision to treat the drinking water with chlorine. Although this subject is beyond the scope of this report, the data collected allows for ample opportunity for further research.