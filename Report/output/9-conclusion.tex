\section{Final remarks}\label{sec:conclusion}

This study aimed to estimate the take-up of Evidence Action’s Dispensers for Safe Water (DSW) and In-line Chlorination (ILC) programs. Using village-level censuses, household surveys, and water point inspections, we estimate that approximately 3.8 million people in Uganda and 2.2 million people in Malawi rely on water points with DSW/ILC as their primary source of drinking water.

Adoption rates, as measured by the presence of Total Chlorine Residual (TCR) in household water samples, range between 27\% and 33\% across village groups in both countries. While this represents a meaningful level of protection, it also underscores the challenge of ensuring consistent take-up and sustained use of chlorinated water at the household level. The data indicates that these challenges are different in DSW and ILC systems.

When DSW dispensers are functional and filled, they are generally dosing at the recommended levels, suggesting that gaps in adoption of this program are related to the reliability of refilling among promoters and to household preferences and behaviors. The functionality rate of water points with ILC devices is low when compared to water points with DSW (60\% vs 98\%\footnote{Table \ref{tab:dsw-func}.}). This may stem from the greater complexity of ILC infrastructure and devices. The fact that FCR detection rates are considerably lower than TCR detection rates indicates that although chlorine was applied to most of the clusters tested, the dosing  was not sufficient for FCR to be detectable given the chlorine demand in the water and tank. The different detection rates in the water collection points closer and further from the tank with the device, seems to point to additional chlorine demand in the pipeline or to chlorine decay as the water is distributed through the pipes, so not all the users of a cluster drink similarly safe water. 

Using the water point or the village as the unit of reference resulted in similar estimates. This may indicate that a water point-based data collection, which we expect to be easier and cheaper than a census-based one, may be a viable solution to measuring reach and adoption.  The challenge remains, however, of also identifying the households using the water sources with DSW or ILC with a less resource-consuming process than a census. The data from this study may help determine an appropriate catchment radius to be used in such a model.

Following Evidence Action’s Adoption Monitoring methods, on the other ahnd, yielded higher estimates of the number of people using water points with DSW and ILC in Malawi than our preferred method, and higher chlorine detection rates in Uganda. Chlorine detection rates from Evidence Action's DSW Q3 Adoption Monitoring are also higher than those obtained using a village-level reference. 

\textcolor{red}{summarise why we think this is}

Two other questions raised at the onset of this study were not discussed or discussed to a lesser extent. 
\textcolor{red}{say something about colorimeter vs color wheel here}

With regard to the other question raised in this study’s concept note, about factors that influence the choice of water source and the decision to treat the drinking water with chlorine, although it is beyond the scope of this report, the data collected allows for ample opportunity for further research. \textcolor{red}{Future exploration of this data could help define a threshold for radius around a water point for future adoption monitoring.}
