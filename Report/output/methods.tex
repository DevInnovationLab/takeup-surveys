\section{Estimation methods}

\subsection{Number of households served by Evidence Action}\label{sec:methods-people}

The number of people served by Evidence Action is calculated based on the number of people living in households whose self-reported primary source of drinking water has a DSW dispenser or is connected to ILC. Standard errors are clustered by village, and group sample weights are based on the population of villages and water points in each country.

We use the R package \textit{survey} to calculate the country-level estimates and their respective confidence interval.\footnote{\cite{survey}} The point estimates are equivalent to the procedure described below, but the package also calculates the standard errors clustered by village and corrected for the sampling weights:

\begin{enumerate}
    \item \textbf{Restrict the household census data to households using water points with DSW or ILC.} 
    \item \textbf{In each unit of reference (village, water point), add the number of people living in households that use a water point with DSW or ILC as their primary source of drinking water.} Due to a survey coding error, the question on the number of household members was skipped during the first two weeks of data collection. In Malawi, this only affected two households, and we impute the number of household members using the village average. In Uganda, there are eight villages with no observations for this variable, and we impute using the sample group average.
    \item \textbf{Take the sample group and intervention average and multiply by the number of villages in this group across the country.} 
\end{enumerate}

\noindent Our main estimates also include the following adjustments: 
\begin{itemize}
    \item To avoid counting the same water point in two villages, we further restrict the sample in Step 1 to households using water points with DSW or ILC located within the village boundaries. Our sensitivity checks include estimates where households using water points located outside of the village boundaries are included.
    \item The sample in Step 1 includes households living up to 200m outside of the village boundaries. We also report results that exclude these households in the sensitivity checks.
    \item We use the constructed, quality-checked version of the intervention variable as the criterion for considering a household to have access to ILC or DSW in Step 1. If a household wasn’t matched to a water point in the census or was recorded as a non-response, it is considered to \textit{not} be using a water point with DSW or ILC. We also report results that use the raw field officer report.
    \item We account for non-responses by assuming that that share of users that did not respond to the household census and use a water point with DSW or ILC is that same as among those that did response. Therefore, between Step 2 and Step 3, we divide the village-level estimate by the response rate. We also report results that do not make this correction.
    \item Villages with no users will have no observations in the household census. To take them into account, we weigh the average calculated in Step 3 by the number of villages in each sample group included in the data obtained in step 2 divided by the number of sampled villages in this group. We also report results that do not include this correction. 
\end{itemize}

Since our unit of sampling is the village, our main estimates use the village as the unit of reference in Step 2. We also report estimates using the water point as the unit of reference, since those may be more comparable to Evidence Action's Adoption Monitoring estimates.

\subsection{Chlorine treatment adoption rates among households served by Evidence Action}

The chlorine adoption rate is estimated by calculating the share of households using a DSW or ILC water point that had chlorine detected in the water sample tested during the household survey.

For the main analysis, we define chlorine detection as meeting the 0.2 mg/L threshold for Total Chlorine Residual, measured using the color wheel. We use the color wheel for the reasons discussed in Section \ref{sec:data-cl} and so our measures are more comparable with those taken by Evidence. The detection threshold is chosen for two main reasons. First, we focus on TCR because, unlike FCR, it decays more slowly and therefore better reflects whether chlorine was applied to the water. It is worth noting, however, that most studies of health effects of water treatment focus on FCR measures.\footnote{Table \ref{tab:thresholds} presents the criteria used by the studies included in \textcite{kremer2023water}.} Second, for the threshold, we adopt 0.2 mg/L because our primary analysis relies on the color wheel, and this value represents the minimum detection threshold of the instrument. Tables \ shows the chlorination rates under different criteria.

We restrict the sample used in our main estimates to households whose self-reported primary source of drinking water was matched to a water point confirmed to have DSW or ILC in the quality-checked version of this variable. That means that not all households in the household survey are included. Our sensitivity checks include estimates that use the raw field officer report of DSW or ILC presence, estimates that restrict the sample to households that self-report that their primary source of drinking water is served by DSW or ILC, and that don't restrict the sample at all, including all households in the household survey.

Our sample also restricts the household sample to those that had water available in the household at the time of the survey. This is a deviation from Evidence Action's Adoption Monitoring protocol, since they assume these households do not chlorinate. Our sensitivity checks include this version of the estimates.

Finally, our main results are based on an unweighted average. That is, every household in the sample has the same weight. We use this estimate because as far as we understand, this is the procedure followed by Evidence Action in their Adoption Monitoring. However, there are a number of different ways to weigh observations that one could argue for. Since are sampling unit is the village, it could be argued that each sampled village, and not household should have the same weight. These two estimates would be similar, since a constant number of households was sampled in each village, but failed tests, attrition, and other survey implementation issues lead the exact number of tests per village to vary. If, however, the goal of this indicator is to present information about the share of all served households in the country that treat their drinking water with chlorine, then villages should be weighed by the number of people using water points with DSW or ILC living in them. Both of these estimates are presented as part of our sensitivity checks.

\subsubsection{Approximating the Adoption Monitoring methods}

When using the promoter survey and the households sampled from the promoter list to calculate the outcomes discussed in the previous sections, a few adaptations from the methods described as necessary.

To estimate the number of people with access to DSW or ILC, we first estimated the average number of people per water point by multiplying the average number of households listed by the promoter at each water point by the average household size among households surveyed from the promoter list. When doing this, we grouped water points into 4 categories: Footprint DSW, Expansion DSW, DSW in ILC villages, and ILC water points (the latter two only apply to Malawi). To estimate the number of people using water points served by Evidence Action at the country level, we calculated the number of water sources of each of these types in each country and multiplied the water point-level estimates by this number. 

We estimated chlorination rates by finding the unweighted average across households in the Monitoring Survey of variables representing whether detected Free Chlorine Residual or Total Chlorine Residual were above 0.2 ppm. Here, we immediately combined all DSW water points, as opposed to first finding chlorination rates separately for Expansion, Footprint, and DSW in ILC villages. This was because the differences in chlorination rates among these categories were found to not be statistically different. 

For chlorination rates, standard errors were estimated accounting for village-level clustering and sample group stratification. Standard errors for household counts and household size were estimated accounting for clustering at the village level and stratification by sample group. For the estimated number of people per water point, standard errors were calculated using the delta method, assuming independence between household counts and household size. Population-level estimates were obtained by scaling water point-level estimates by the country number of water points of each category, with standard errors scaled proportionally. Country totals across categories were aggregated assuming independence, with standard errors estimated as the square root of the sum of squared standard errors.
