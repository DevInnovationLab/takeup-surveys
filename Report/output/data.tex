\section{Data}

This Section describes the final sample sizes obtained the different survey instruments described in Section \ref{sec:design-instruments}. It also indicates deviations from the original survey protocol, and flags potential sources of bias.

\subsection{Administrative data from Evidence Action}

Our sample was selected based on administrative data received from Evidence Action. This database lists all the water points installed by Evidence Action, as well as their IDs, GPS coordinates, installation dates, the type of water source, and information on the administrative unit where they are located.\footnote{The last version of this database was shared on May 22, 2025.} It also contains similar information on the water collection points connected to ILC water points.

Since our unit of reference is the village, we conducted extensive data processing to harmonize the village names and other administrative unit information. This data seems to have originally been entered manually, with different spellings used by different enumerators, as well as different assignments of higher-level administrative units (such as group village head, parish, and district) to the same village. We relied mostly on names and GPS coordinates to do this. In Uganda, we also used a village-level shapefile from \textcite{ssentongo2018changes}.

\noindent In Malawi, this database covers covers:
\begin{itemize}
    \item 6,607 villages in 10 districts
    \item 16,198 water points with DSW across 6,245 villages in 8 districts
    \item 352 water points with an ILC device across 285 villages in 10 districts;
    \item 2,024 water collection points across 626 villages in 10 districts.
\end{itemize}

\noindent In Uganda, it covers:
\begin{itemize}
    \item 7,834 villages in 22 districts
    \item 17,718 water points with DSW across 7,416 villages in 22 districts
\end{itemize}

\subsection{Uganda sample}\label{sec:data-uganda}

\begin{itemize}
\item 64 villages were visited.
  \begin{itemize}
  \item Four were replaced, one because no dispensers were found in it, and
    the other three because the field team observed upon arrival that
    the chairpeople or the promoters had been informed about the surveys
    in advance.
  \item 60 are included in our analysis, 30 from the Footprint sample group, and 30 from the Expansion sample group.
  \end{itemize}
\item In these 60 villages, 524 water points were identified during the census.
  \begin{itemize}
  \item 452 were used for drinking by more than one household, and therefore
    eligible to be included in the water point census.
  \item 11 water points from 6 distinct villages were served by ILC. These
    water points are excluded from our analysis.
  \end{itemize}
\item 185 water points with DSW were identified.\footnote{\textcolor{red}{The correspondence between water point census IDs and Evidence Action IDs can be found in \href{https://docs.google.com/spreadsheets/d/1Nb_ZgIIEIxLdFtRAb-BdVcoQGqHqXLKBGduU_um9BeA/edit?pli=1&gid=1590291999#gid=1590291999}{this document}.}}
  \begin{itemize}
  \item 172 were matched to the Evidence Action database using dispenser IDs
    and GPS locations.
  \item 13 could not be matched to water points in the Evidence Action data. Ten had pictures available from the water point census which confirmed the presence of a dispenser,\footnote{The bar codes and other identifying information on these water points were shared with Evidence Action.} and three were non-functional water points and data on dispenser identification was not collected.
  \item \textcolor{red}{11 water points with DSW included in the Evidence Action data could not be matched to water points with DSW in the census}. Four were located by the field team, but no longer had dispensers, and seven were outside of the village boundaries according to the GPS data.
  \end{itemize}
\item Roughly 9.5 thousand households were listed in visited villages, and 8.5 thousand of them (88.9\%) were surveyed as part of the household census.
\item 6 thousand (75.4\%) of the households in the census used water points with DSW as their primary source of drinking water, and 1,209 completed the household survey.
  \begin{itemize}
  \item 1,133 of the household in the household census (93.7\%) were using water points that were confirmed to have DSW. The remaining households either used water points that were either re-classified as Non-program during data quality checks (2.1\%) or could not be matched to the water point census (4.2\%). 
  \item 1,067 (94.2\%) of the households surveyed had drinking water available
    at the time of the survey, and 1,059 valid color wheel tests were conducted.\footnote{For a description of the data quality checks conducted on chlorine tests and the reasons for removing tests from the data, see Section \ref{sec:design-chlorine}.}
  \end{itemize}
\item
  \textcolor{red}{117 promoter surveys were conducted}, representing
  63.2\% of the water points with DSW identified. The remaining water points are not included in estimates based on the promoter sample.
  \begin{itemize}
      \item Evidence Action provided information for promoters responsible for \textcolor{red}{118} unique water points. All of these water points are included in the water point census, and promoter surveys were conducted in \textcolor{red}{110} of them (\textcolor{red}{84.7\%}).
      \item The field team also identified another \textcolor{red}{7} promoters
        for water points that were not included the Evidence Action contact
        list.
      \item \textcolor{red}{1} of the promoters was responsible for a water
        points that is no longer being served by DSW. \textcolor{red}{Observations linked to this water point is not included in our analysis.}
  \end{itemize}
\item 460 households listed by promoters as using water points with ILC
  completed the household survey.
  \begin{itemize}
  \item 400 valid color wheel tests are observed. The vast majority of households with missing test observations did not have water available in the household at the time of the survey.
  \end{itemize}
\end{itemize}

\subsection{Malawi sample}\label{sec:data-malawi}

\begin{itemize}
    \item 108 villages were visited. 
    \begin{itemize}
        \item Our original sample included 100 villages: 30 from the Footprint sample group, 30 from the Expansion sample group, and 41 from the ILC sample group. In the ILC sample group, 14 sampled villages were served by both ILC and DSW.
        \item Three villages were replaced because boundaries were under dispute or the villages could not be found based on the information from the Evidence Action database
        \item One village was replaced because the water point with DSW was only used by a mosque and most households had municipality water supply.
        \item Three ILC villages were replaced because all of the water collection points were out of order at the time of the visit, and a fourth one because the village guide reported that the village was not served by ILC.
        \item Given the distance between villages, how difficult transportation with the country is, and how often replacements were needed, the protocols for replacing villages had to be adjusted. As a result, if a village was served by both ILC and DSW and at least one water points with DSW was functional and eligible for inclusion in the survey but no water points with ILC satisfied these conditions, the village was not replaced. Additionally, four villages were replaced by villages in a different sample group.
        \end{itemize}
    \item 100 villages are included in our final data: 28 from the Footprint sample group, 28 from the Expansion sample group, and 44 from the ILC sample group.
    \begin{itemize}
        \item In the ILC sample group, the water points census indicates 37 villages have water collection points connected to ILC and 24 villages had water points with DSW.
    \end{itemize}
    \item 202 DSW water points were identified during the water point census 
    \begin{itemize}
        \item 66 were located in villages from the ILC sample group
        \item 197 were matched to water points with DSW in the Evidence Action data, and remaining 5 were verified to have dispensers using photos.
        \item \textcolor{red}{47 water points with DSW mapped to sampled villages in the Evidence Action could not be matched to water points with DSW in our data}. Out of these, 31 were located outside of the sampled villages, 13 had no dispensers installed at the time of the visit, 2 were duplicate entries for the same water point, and 1 was found to have ILC and not DSW.
    \end{itemize}
    \item 26 water points that either had an ILC device installed at the time of the visit or used to have one in the past were identified during the water point census, as well as 176 water collection points connected to a tank with ILC.
    \begin{itemize}
        \item \textcolor{red}{The adimnistrative data we received suggested we should find 30 water points with an ILC device and 193 water collection points connected to them in the 38 villages included in our ILC sample.}
        \item \textcolor{red}{TX} ILC water points feeding these water collection points were not mapped because they were located very far from the village.
        \item Cross-referencing our data with the administrative data using GPS coordinates, we are able to identify \textcolor{red}{31} clusters with one of more water collection points in the villages sampled. Most clusters consist of one water point with an ILC device and one or two water collection points connected to it. However, larger clusters may have up to 34 water collection points present in our data across up to five different villages. 
        \item A total of 60 water collection points had a water sample tested for chlorine residual using a color wheel, and 58 using a colorimeter.
        \item \textcolor{red}{ In total, 22 clusters were tested. For the largest cluster in our sample, we have tested water samples from 11 different water collection points.}
    \end{itemize}
    \item Roughly 11.5 thousand households were listed in the household census
    \begin{itemize}
        \item 11.3 thousand households (98.1\%) responded to the household census\footnote{This may seem an unusually high response rate, but the field team repeatedly confirmed that there were very few non-responses.}
        \item 5.3 thousand households (47\% of the households that answered the census questions) were using water points with DSW, and 1.2 thousand (10.7\%) were using water points with ILC.
    \end{itemize}
    \item 1,967 households in 99 villages completed the household survey.\footnote{In one village, households reported that they do not use the water point with the chlorine dispenser to collect drinking water, due to its high salinity. Therefore, no household surveys were conducted in this village.}
    \begin{itemize}
        \item 1,952 had drinking water available at the household during the survey, and the data includes 1,913 valid color wheel test
    \end{itemize}
\end{itemize}

\subsection{Descriptive statistics}

\noindent\textbf{Average village sizes.} The number of households living in sampled villages in Malawi are comparable across all village groups, ranging from 111.4 in Expansion villages to 122.5 in Footprint villages. In Uganda, on the other hand, Expansion villages are considerably larger, with on average 203.5 households living in them, than Footprint villages, with an average of 114 households per village.

\textcolor{red}{households living within the village}

\noindent\textbf{Use of water points served by Evidence Action.} Around 60\% of the households living in sampled villages reported using a primary water source served by Evidence Action, with considerable heterogeneity across village groups (Table 5, column (5)). The average share of households whose primary source of drinking water has DSW or ILC ranges from 35\% to 69\%, depending on the village type. The highest rate is observed among DSW Expansion villages in Malawi (69.1\%), while the lowest is recorded in Malawi ILC villages (35.2\%).

\noindent\textbf{Number of water points used.} The majority of households—about 80 percent—rely on a single water point as their primary source, while the remaining 20 percent report using more than one. We find no significant differences in the prevalence of secondary water sources across village categories, which range from 74\% in footprint villages in Uganda to 85\% in expansion villages of Malawi. Likewise, the number of water sources used does not vary between households relying on DSW, ILC, or non-EA sources.

\noindent\textbf{Average household size.} Households in Uganda tend to be larger than those in Malawi, with an average of 5.4 people living in a household in the former and 4.5 in the latter. Household size is comparable in household using water points with DSW and ILC and those who aren’t in households from the Expansion and Footprint village groups, but in ILC villages, households using water points with ILC have on average 0.5 less members. Due to a coding problem in our survey, the question on the number of people living in a household was only introduced around three weeks into data collection. Due to these two factors, we impute the number of users in villages where these information is missing using the village group average among households that use water points with DSW or ILC when calculating the number of people using water points with DSW or ILC.

\noindent\textbf{Source of water sample tested:} Water samples for chlorine testing were drawn mainly from the households' primary water point, with the share of such cases ranging from 85\% in Uganda Footprint villages to 96\% in Malawi Expansion villages. A smaller share reported collecting water from another point inside the village (2–9\%), from outside the village (2–4\%), or from rainwater or piped sources (each under 3\%). These patterns indicate that the testing data largely reflect the water sources households rely on most.

\subsection{Discussion}\label{sec:data-bias}

\textbf{Unseasonal rains occurred in Uganda during data collection, which could lead to an underestimation of chlorine adoption.} Data collection in Uganda and Malawi took place between July and August 2025. To avoid bias from increased use of untreated rainwater during the rainy season, the timing of the surveys was planned to avoid the peak rainy periods.\footnote{In Uganda, there are two rainy seasons: a main season from March to May and a shorter one from September to November. The data collection was scheduled shortly after the first rainy season. In Malawi, the country experiences a single, extended rainy season from November to April, with the heaviest rainfall occurring between January and March. The dry season follows, with a cool dry period from May to August and a hot dry period from September to October.} However, heavy rains occurred during data collection in 14 villages in Uganda, and lighter rains were reported in 6 others, meaning households in these villages may have been relying on rainwater more than regularly. Although chlorination rates in villages were rains occurred are not consistently different from those observed in villages were it did not rain, we include sensitivity checks that exclude the first group from chlorination rates estimates.  

\textbf{In a few villages, we identified cases in Uganda where a household was not listed in the household census} because it could not be surveyed after three attempts and no neighbors were found to ask for the name of the head of the household, even though the survey protocol dictated that these households be recorded as non-responses. Therefore, we may underestimate the number of households living in sampled villages and using water points with DSW or ILC.

\textbf{In Malawi, the final numbers of villages per group deviate slightly from the original targets due to the reasons discussed in Section \ref{sec:data-malawi}.} This deviation has two main implications. First, the number of villages served by ILC is smaller than originally planned, which adds noise to our estimates and increases the risk of small sample bias. Second, the share of villages served by DSW where ILC is available or was available at some point in time is larger than originally designed.

\textbf{In villages served by both DSW and ILC, the presence of ILC could lead to lower chlorination rates even when households use DSW water sources.} This could happen, for example, because households only use DSW water points when ILC water points are not functional, and therefore do not develop the habit of using the dispenser, or because they expect the water to already be treated, as is the case in ILC water points. Since the chlorination rates are calculated using the simple average of tested households, this could create a downward bias in chlorination rates among households using water points with DSW. We adopt two different strategies to address this issue: we report chlorination rates among households using water points with DSW separately for each sample group, and we present an alternatively DSW-wide estimate that weights village-level chlorination rates based on the number of water points in each sample group.

\textbf{Survey protocols dictated that teams should not contact the village leaders until they arrived at the village, and the promoters until the time of the promoter survey, to avoid priming.} Evidence Action also instructed local teams to not inform their village contacts about the survey activities, and keeping all communications restricted to usual operations. Still, in three villages in Uganda, the field team observed upon arrival that the chairperson and the local promoters had prepared for the survey, filling dispensers and reminding households to use chlorination. These three villages were replaced. In five other villages, the field team only learned that promoters and households had prepared for the survey at an advanced stage of the activities in the village. These villages were not replaced, but we conducted sensitivity checks and found that their inclusion does not meaningfully affect results. Therefore, these villages are included in our main estimates. Evidence Action was informed of these incidents and responded by reinforcing the message to avoid communications that could interfere with the data collection.

\textbf{A number of water points in Evidence Action database in Malawi were located outside of the boundaries of the villages where they should be located, which could lead to underestimation of the population reached.} Water points that the Evidence Action data indicated were in the same village turned out to be in different villages. If villages were split and this is not reflected in the Evidence Action data, or if the wrong village name was assigned, for example by replacing it with the group village head, then the number of villages used to scale estimated from our sample to the country underestimates Evidence Action's coverage, and therefore the number of people reached. \textcolor{red}{If that was the case, then we would see different estimates at the water point level, and we also looked at the data.}
